\documentclass{article}
\usepackage{mathtext}
\usepackage[russian]{babel}
\usepackage[a4paper, paperheight=14cm, top=1cm,bottom=1cm,left=1cm,right=1cm,marginparwidth=1.75cm]{geometry}
\usepackage{amsmath}
\usepackage{amssymb}
\usepackage{graphicx}
\usepackage{cancel}
\usepackage{wrapfig}
\pagenumbering{gobble}

\begin{document}
\noindent \begin{tabular}{l|l}
    Дано: & Траектория движения тела, брошенного под углом параболическое, т.к. на него действует \\
    $m = 12$ кг & ускорение свободного падения. В верхней точки $R$ вектор скорости параллелен оси ox \\
    $\varphi = 45^{\circ}$ & и равен $V_в = V_0\cos\varphi$. \\
    $V_0 = 200$ м/с & \\
    $V_1 = 400$ м/с & 1. Спроецируем закон сохранения импульса $mv = \sum m_iv_i$ на каждую из осей: \\ 
    $V_2 = 100$ м/с & ox: $mV_в = \frac{m}{3}(V_1\cos\varphi+0+V_{3x})$ \\
    \rule{58px}{0.5pt} & oy: $0 = \frac{m}{3}(V_1\sin\varphi + V_2 + V_{3y})$\\
    $V_3$ --- ? & \\
\end{tabular} \begin{wrapfigure}{r}{0.45\textwidth}
    \includegraphics[width=\linewidth]{"3_graphics"}
    \end{wrapfigure} \\ \, \\
\noindent 2. Подставим значения в формулу \\
\noindent ox: $\cancelto{3}{12}\cdot200\frac{\sqrt{2}}{2} = \cancel{4}(400\frac{\sqrt{2}}{2}+0+V_{3x})$ \\
\noindent oy: $0 = \cancel{4}(400\frac{\sqrt{2}}{2}+100+V_{3y})$ \\ \, \\
\noindent 3. Выразим искомую $V_3$ в обеих проекциях: \\
\noindent ox: $300\sqrt{2} - 200\sqrt{2} = V_{3x} \Leftrightarrow 100\sqrt{2} = V_{3x}$ \\
\noindent oy: $-200\sqrt{2} - 100 = V_{3y} \Leftrightarrow -100(2\sqrt{2}-1) = V_{3y}$ \\ \, \\
\noindent 4. Найдём исходную скорость как гипотенузу, равную сумме квадратов проекций как катетов: \\
\noindent $V_3 = \sqrt{V_{3x}^2 + V_{3y}^2} \approx \sqrt{2000 + 146568.54} \approx 385.45$

\begin{flushright}
Ответ: 385.45 м/с
\end{flushright}

\end{document}