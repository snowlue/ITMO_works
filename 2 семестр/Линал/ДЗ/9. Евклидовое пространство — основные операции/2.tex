\documentclass{article}
\usepackage{cmap}
\usepackage{mathtext}
\usepackage[russian]{babel}
\usepackage[a3paper, paperwidth=28cm, paperheight=14cm, top=1cm,bottom=1cm,left=1cm,right=1cm,marginparwidth=1.75cm]{geometry}
\usepackage{amsmath}
\usepackage{amssymb}
\usepackage{multicol}
\usepackage{fancyhdr}
\usepackage{nicefrac}
\usepackage{graphicx}
\usepackage{cancel}
\usepackage{wrapfig}
\usepackage{tikz}
\pagenumbering{gobble}


\newlength{\tempheight}
\newcommand{\Let}[0]{%
\mathbin{\text{\settoheight{\tempheight}{\mathstrut}\raisebox{0.5\pgflinewidth}{%
\tikz[baseline,line cap=round,line join=round] \draw (0,0) --++ (0.4em,0) --++ (0,1.5ex) --++ (-0.4em,0);%
}}}\;}
\newcommand{\e}{\text{e}}
\newcommand{\la}{\lambda}
\newcommand{\shiftleft}[3]{\makebox[#1][r]{\makebox[#2][l]{#3}}}
\newcommand{\shiftright}[3]{\makebox[#2][r]{\makebox[#1][l]{#3}}}
\newcommand*\circled[1]{\tikz[baseline=(char.base)]{
            \node[shape=circle,draw,inner sep=2pt] (char) {#1};}}
\newcommand*\squared[1]{\tikz[baseline=(char.base)]{
            \node[shape=rectangle,draw,inner sep=4pt] (char) {#1};}}
\newcommand{\at}{\biggr\rvert}

\begin{document}
\begin{center}
    Обозначим $x_1$ и $x_2$ векторы внутри линейной оболочки и найдём $z \subset L$
    $$\begin{cases}
            (x_1, v) = \alpha_1 (x_1, x_1) + \alpha_2 (x_1, x_2) \\
            (x_2, v) = \alpha_1 (x_2, x_1) + \alpha_2 (x_2, x_2)
        \end{cases} \Leftrightarrow \left( \begin{array}{cc|c}
                (x_1, x_1) & (x_1, x_2) & (x_1, v) \\
                (x_2, x_1) & (x_2, x_2) & (x_2, v)
            \end{array} \right) \Leftrightarrow \left( \begin{array}{cc|c}
                9+36+81    & -18-90-189 & -9-54-135  \\
                -18-90-189 & 36+225+441 & 18+135+315
            \end{array} \right) \Leftrightarrow \left( \begin{array}{cc|c}
                126  & -297 & -198 \\
                -297 & 702  & 468
            \end{array} \right) \sim$$
    $$\sim \left( \begin{array}{cc|c}
                14  & -33 & -22 \\
                -11 & 78  & 52
            \end{array} \right) \sim \left( \begin{array}{cc|c}
                14 & -33 & -22 \\
                3  & 45  & 30
            \end{array} \right) \sim \left( \begin{array}{cc|c}
                14 & -33 & -22 \\
                1  & 15  & 10
            \end{array} \right) \sim \left( \begin{array}{cc|c}
                14 & -33 & -22 \\
                14 & 210 & 140
            \end{array} \right) \sim \left( \begin{array}{cc|c}
                14 & -33 & -22 \\
                0  & 243 & 162
            \end{array} \right) \sim \left( \begin{array}{cc|c}
                14 & -33 & -22 \\
                0  & 3   & 2
            \end{array} \right) \sim \left( \begin{array}{cc|c}
                14 & 0 & 0 \\
                0  & 3 & 2
            \end{array} \right) \Rightarrow$$
    $$\Rightarrow \begin{cases}
            \alpha_1 = 0 \\
            \alpha_2 = \frac{2}{3}
        \end{cases} \Rightarrow z = \frac{2}{3}x_2 = \begin{pmatrix}
            4 \\ -10 \\ -14
        \end{pmatrix}$$
    Задача сводится к нахождению угла между $v$ и $z$:
    $$\cos\varphi = \frac{|z^2|}{|v^2|} = \frac{\sqrt{16+100+196}}{\sqrt{9+81+225}} = \frac{2\sqrt{78}}{3\sqrt{35}} \Rightarrow \varphi = \arccos\frac{2\sqrt{78}}{3\sqrt{35}} \approx 0.1$$
\end{center}
\end{document}