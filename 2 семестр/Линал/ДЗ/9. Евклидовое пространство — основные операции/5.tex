\documentclass{article}
\usepackage{cmap}
\usepackage{mathtext}
\usepackage[russian]{babel}
\usepackage[a3paper, paperwidth=28cm, paperheight=14cm, top=1cm,bottom=1cm,left=1cm,right=1cm,marginparwidth=1.75cm]{geometry}
\usepackage{amsmath}
\usepackage{amssymb}
\usepackage{multicol}
\usepackage{fancyhdr}
\usepackage{nicefrac}
\usepackage{graphicx}
\usepackage{cancel}
\usepackage{wrapfig}
\usepackage{tikz}
\pagenumbering{gobble}


\newlength{\tempheight}
\newcommand{\Let}[0]{%
\mathbin{\text{\settoheight{\tempheight}{\mathstrut}\raisebox{0.5\pgflinewidth}{%
\tikz[baseline,line cap=round,line join=round] \draw (0,0) --++ (0.4em,0) --++ (0,1.5ex) --++ (-0.4em,0);%
}}}\;}
\newcommand{\e}{\text{e}}
\newcommand{\la}{\lambda}
\newcommand{\shiftleft}[3]{\makebox[#1][r]{\makebox[#2][l]{#3}}}
\newcommand{\shiftright}[3]{\makebox[#2][r]{\makebox[#1][l]{#3}}}
\newcommand*\circled[1]{\tikz[baseline=(char.base)]{
            \node[shape=circle,draw,inner sep=2pt] (char) {#1};}}
\newcommand*\squared[1]{\tikz[baseline=(char.base)]{
            \node[shape=rectangle,draw,inner sep=4pt] (char) {#1};}}
\newcommand{\at}{\biggr\rvert}

\begin{document}
\begin{center}
    Для того, чтобы найти угол между векторами, для начала нужно найти косинус этого угла, который вычисляется по следующей формуле:
    $$\cos\varphi = \frac{x^{T}Gy}{\sqrt{x^{T}Gx}\sqrt{y^{T}Gy}}$$
    Разделим на компоненты и посчитаем каждый отдельно:
    $$x^{T}Gy = \begin{pmatrix}
            -3 & 6
        \end{pmatrix}\times\begin{pmatrix}
            3 & 4 \\ 4 & 6
        \end{pmatrix}\times\begin{pmatrix}
            3 \\ -3
        \end{pmatrix} = \begin{pmatrix}
            15 & 24
        \end{pmatrix} \times \begin{pmatrix}
            3 \\ -3
        \end{pmatrix} = -27$$
    $$x^{T}Gx = \begin{pmatrix}
            -3 & 6
        \end{pmatrix}\times\begin{pmatrix}
            3 & 4 \\ 4 & 6
        \end{pmatrix}\times\begin{pmatrix}
            -3 \\ 6
        \end{pmatrix} = \begin{pmatrix}
            15 & 24
        \end{pmatrix} \times \begin{pmatrix}
            -3 \\ 6
        \end{pmatrix} = 99$$
    $$y^{T}Gy = \begin{pmatrix}
            3 & -3
        \end{pmatrix}\times\begin{pmatrix}
            3 & 4 \\ 4 & 6
        \end{pmatrix}\times\begin{pmatrix}
            3 \\ -3
        \end{pmatrix} = \begin{pmatrix}
            -3 & -6
        \end{pmatrix} \times \begin{pmatrix}
            3 \\ -3
        \end{pmatrix} = 9$$
    Посчитаем косинус, подставив найденные выше значения:
    $$\cos\varphi = \frac{-27}{\sqrt{99}\sqrt{9}} = -\frac{3\sqrt{11}}{11} \Rightarrow \varphi = \pi - \arccos\frac{3\sqrt{11}}{11} \approx 2.70$$
\end{center}
\end{document}