\documentclass{article}
\usepackage{mathtext}
\usepackage[russian]{babel}
\usepackage[a3paper, paperwidth=28cm, paperheight=14cm, top=1cm,bottom=1cm,left=1cm,right=1cm,marginparwidth=1.75cm]{geometry}
\usepackage{amsmath}
\usepackage{amssymb}
\usepackage{multicol}
\usepackage{fancyhdr}
\usepackage{nicefrac}
\usepackage{graphicx}
\usepackage{cancel}
\usepackage{wrapfig}
\usepackage{tikz}
\pagenumbering{gobble}


\newlength{\tempheight}
\newcommand{\Let}[0]{%
\mathbin{\text{\settoheight{\tempheight}{\mathstrut}\raisebox{0.5\pgflinewidth}{%
\tikz[baseline,line cap=round,line join=round] \draw (0,0) --++ (0.4em,0) --++ (0,1.5ex) --++ (-0.4em,0);%
}}}\;}
\newcommand{\e}{\text{e}}
\newcommand{\la}{\lambda}
\newcommand{\shiftleft}[3]{\makebox[#1][r]{\makebox[#2][l]{#3}}}
\newcommand{\shiftright}[3]{\makebox[#2][r]{\makebox[#1][l]{#3}}}
\newcommand*\circled[1]{\tikz[baseline=(char.base)]{
            \node[shape=circle,draw,inner sep=2pt] (char) {#1};}}
\newcommand*\squared[1]{\tikz[baseline=(char.base)]{
            \node[shape=rectangle,draw,inner sep=4pt] (char) {#1};}}
\newcommand{\at}{\biggr\rvert}

\begin{document}
\begin{center}
Определим свёртку по $i$ как $b^{pk}_p = a^{1pk}_{1p} + a^{2pk}_{2p}$. \\
Тогда свёртка по $p$ будет выглядеть как $c^k = b^{1k}_1 + b^{2k}_2 = \left (a^{11k}_{11} + a^{21k}_{21}\right ) + \left (a^{12k}_{12} + a^{22k}_{22}\right )$. \\
Отобразим свёртку в тензорном виде:
$$c^k = \begin{Vmatrix}
a^{111}_{11} \\ \, \\ a^{112}_{11}
\end{Vmatrix} + \begin{Vmatrix}
a^{211}_{21} \\ \, \\ a^{212}_{21}
\end{Vmatrix} + \begin{Vmatrix}
a^{121}_{12} \\ \, \\ a^{122}_{12}
\end{Vmatrix} + \begin{Vmatrix}
a^{221}_{22} \\ \, \\ a^{222}_{22}
\end{Vmatrix}$$
Сопоставим каждому из компонентов соответствующее соотношение $4j + k - 4i - 4p$ и получим:
$$c^k = \begin{Vmatrix}
-3 \\ -2
\end{Vmatrix} + \begin{Vmatrix}
-3 \\ -2
\end{Vmatrix} + \begin{Vmatrix}
-7 \\ -6
\end{Vmatrix} + \begin{Vmatrix}
-7 \\ -6
\end{Vmatrix} = \begin{Vmatrix}
-6 \\ -4
\end{Vmatrix} + \begin{Vmatrix}
-14 \\ -12
\end{Vmatrix} = \begin{Vmatrix}
-20 \\ -16
\end{Vmatrix}$$ 
\end{center}
\end{document}