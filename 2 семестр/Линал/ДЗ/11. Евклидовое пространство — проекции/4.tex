\documentclass{article}
\usepackage{cmap}
\usepackage{mathtext}
\usepackage[russian]{babel}
\usepackage[a3paper, paperwidth=28cm, paperheight=14cm, top=1cm,bottom=1cm,left=1cm,right=1cm,marginparwidth=1.75cm]{geometry}
\usepackage{amsmath}
\usepackage{amssymb}
\usepackage{multicol}
\usepackage{fancyhdr}
\usepackage{nicefrac}
\usepackage{graphicx}
\usepackage{cancel}
\usepackage{wrapfig}
\usepackage{tikz}
\pagenumbering{gobble}


\newlength{\tempheight}
\newcommand{\Let}[0]{%
\mathbin{\text{\settoheight{\tempheight}{\mathstrut}\raisebox{0.5\pgflinewidth}{%
\tikz[baseline,line cap=round,line join=round] \draw (0,0) --++ (0.4em,0) --++ (0,1.5ex) --++ (-0.4em,0);%
}}}\;}
\newcommand{\e}{\text{e}}
\newcommand{\la}{\lambda}
\newcommand{\shiftleft}[3]{\makebox[#1][r]{\makebox[#2][l]{#3}}}
\newcommand{\shiftright}[3]{\makebox[#2][r]{\makebox[#1][l]{#3}}}
\newcommand*\circled[1]{\tikz[baseline=(char.base)]{
            \node[shape=circle,draw,inner sep=2pt] (char) {#1};}}
\newcommand*\squared[1]{\tikz[baseline=(char.base)]{
            \node[shape=rectangle,draw,inner sep=4pt] (char) {#1};}}
\newcommand{\at}{\biggr\rvert}

\begin{document}
\begin{center}
    В рамках линейной оболочки понадобится ортогонализировать два вектора:
    $$e_1 = l_1 = \begin{pmatrix}
            -3 \\ 3 \\ -6 \\ 15
        \end{pmatrix}\qquad
        e_2 = l_2 - \frac{e_1^{T}Gl_2}{e_1^TGe_1}e_1 = \begin{pmatrix}
            -3 \\ 3 \\ -3 \\ 9
        \end{pmatrix} - \frac{1998}{3717}\begin{pmatrix}
            -3 \\ 3 \\ -6 \\ 15
        \end{pmatrix} = \begin{pmatrix}
            \nicefrac{-573}{413} \\ \nicefrac{573}{413} \\ \nicefrac{93}{413} \\ \nicefrac{387}{413}
        \end{pmatrix}$$
    $$e_3 = l_3 - \frac{e_2^{T}Gl_3}{e_2^TGe_2}e_2 - \frac{e_1^{T}Gl_3}{e_1^TGe_1}e_1 = \begin{pmatrix}
            0 \\ 6 \\ -6 \\ 15
        \end{pmatrix} + \frac{1358}{551}\begin{pmatrix}
            \nicefrac{-573}{413} \\ \nicefrac{573}{413} \\ \nicefrac{93}{413} \\ \nicefrac{387}{413}
        \end{pmatrix} - \frac{4095}{3717}\begin{pmatrix}
            -3 \\ 3 \\ -6 \\ 15
        \end{pmatrix} = \begin{pmatrix}
            \nicefrac{-63}{551} \\ \nicefrac{3369}{551} \\ \nicefrac{642}{551} \\ \nicefrac{432}{551}
        \end{pmatrix}$$
    Найдём ортогональную проекцию:
    $$
        y_L = \frac{e_1^{T}Gy}{e_1^TGe_1}e_1 + \frac{e_2^{T}Gy}{e_2^TGe_2}e_2 + \frac{e_3^{T}Gy}{e_3^TGe_3}e_3 = \frac{-7992}{3717}\begin{pmatrix}
            -3 \\ 3 \\ -6 \\ 15
        \end{pmatrix} - \frac{\nicefrac{2502}{413}}{551}\begin{pmatrix}
            \nicefrac{-573}{413} \\ \nicefrac{573}{413} \\ \nicefrac{93}{413} \\ \nicefrac{387}{413}
        \end{pmatrix} - \frac{\nicefrac{6714}{551}}{\nicefrac{17460}{551}}\begin{pmatrix}
            \nicefrac{-63}{551} \\ \nicefrac{3369}{551} \\ \nicefrac{642}{551} \\ \nicefrac{432}{551}
        \end{pmatrix} =
    $$
    $$
        = \begin{pmatrix}
            \nicefrac{593439834591}{91164013430} \\ \nicefrac{-803774950113}{91164013430} \\ \nicefrac{567505406253}{45582006715} \\ \nicefrac{-1484314508682}{45582006715}
        \end{pmatrix} = \begin{pmatrix}
            \text{ребят}      \\
            \text{ну это уже} \\
            \text{совсем}     \\
            \text{перебор...}
        \end{pmatrix}
    $$
\end{center}
\end{document}