\documentclass{article}
\usepackage{mathtext}
\usepackage[russian]{babel}
\usepackage[a3paper, paperwidth=28cm, paperheight=14cm, top=1cm,bottom=1cm,left=1cm,right=1cm,marginparwidth=1.75cm]{geometry}
\usepackage{amsmath}
\usepackage{amssymb}
\usepackage{graphicx}
\usepackage{cancel}
\usepackage{wrapfig}
\pagenumbering{gobble}
\usepackage{tikz,amstext}

\newlength{\tempheight}
\newcommand{\Let}[0]{%
\mathbin{\text{\settoheight{\tempheight}{\mathstrut}\raisebox{0.5\pgflinewidth}{%
\tikz[baseline,line cap=round,line join=round] \draw (0,0) --++ (0.4em,0) --++ (0,1.5ex) --++ (-0.4em,0);%
}}}\;}

\begin{document}
\begin{center}
$\Let \varphi$: $x \to y$, при этом $\{e\}$ и $\{\tilde{e}\}$ --- базис $x$, а $\{h\}$ и $\{\tilde{h}\}$ --- базис $y$. Для перехода из между базисами нужны матрицы перехода $T_x$ и $T_y$.\\
Осталось выяснить их расположение в конечной формуле: \\
$\begin{cases}
A\cdot x = y \\
\tilde{A}\cdot \tilde{x} = \tilde{y}
\end{cases} \Rightarrow \left[\begin{array}{l}
\tilde{x} = S_x\cdot  x \\
\tilde{y} = S_y\cdot  y
\end{array}\right] \Rightarrow \begin{cases}
A\cdot x = y \\
\tilde{A} \cdot S_x\cdot x = S_y\cdot y
\end{cases} \Rightarrow \tilde{A} \cdot S_x\cdot \cancel{x} = S_y\cdot A\cdot \cancel{x} \Rightarrow \begin{array}{|c|}\hline \\ [-1ex]\tilde{A} = S_y\cdot A\cdot T_x \\[1.5ex] \hline\end{array}$ \\
Тогда $S_y = \tilde{h}^{-1}h$, а $T_x = e^{-1}\tilde{e}$. Приступим к вычислениям:
$$S_y = \left(\begin{array}{rrr}
1 & 1 & -2 \\ -2 & -1 & 3 \\ -4 & -3 & 8
\end{array}\right)^{-1}\times\left(\begin{array}{rrr}
1 & 1 & 1 \\ 1 & 2 & 0 \\ -2 & -3 & 0
\end{array}\right) = \left(\begin{array}{rrr}
1 & -2 & 1 \\ 4 & 0 & 1 \\ 2 & -1 & 1
\end{array}\right)\times\left(\begin{array}{rrr}
1 & 1 & 1 \\ 1 & 2 & 0 \\ -2 & -3 & 0
\end{array}\right) = \left(\begin{array}{rrr}
-3 & -6 & 1 \\ 2 & 1 & 4 \\ -1 & -3 & 2
\end{array}\right)$$
$$T_x = \left(\begin{array}{rrr}
1 & 1 & -1 \\ -1 & 0 & 2 \\ -2 & -1 & 4
\end{array}\right)^{-1}\times\left(\begin{array}{rrr}
-1 & 1 & -2 \\ 1 & 0 & 1 \\ 3 & -2 & 6
\end{array}\right) = \left(\begin{array}{rrr}
2 & -3 & 2 \\ 0 & 2 & -1 \\ 1 & -1 & 1
\end{array}\right)\times\left(\begin{array}{rrr}
-1 & 1 & -2 \\ 1 & 0 & 1 \\ 3 & -2 & 6
\end{array}\right) = \left(\begin{array}{rrr}
1 & -2 & 5 \\ -1 & 2 & -4 \\ 1 & -1 & 3
\end{array}\right)$$ \\
$$\tilde{A}_x = \left(\begin{array}{rrr}
-3 & -6 & 1 \\ 2 & 1 & 4 \\ -1 & -3 & 2
\end{array}\right) \times \left(\begin{array}{rrr}
15 & -6 & 15 \\ -6 & 3 & -6 \\ 3 & 0 & 0
\end{array}\right) \times \left(\begin{array}{rrr}
1 & -2 & 5 \\ -1 & 2 & -4 \\ 1 & -1 & 3
\end{array}\right) = \left(\begin{array}{rrr}
-6 & 0 & -9 \\ 36 & -9 & 24 \\ 9 & -3 & 3
\end{array}\right) \times \left(\begin{array}{rrr}
1 & -2 & 5 \\ -1 & 2 & -4 \\ 1 & -1 & 3
\end{array}\right) = \left(\begin{array}{rrr}
-15 & 21 & -57 \\ 69 & -114 & 288 \\ 15 & -27 & 66
\end{array}\right)$$
\end{center}
\end{document}