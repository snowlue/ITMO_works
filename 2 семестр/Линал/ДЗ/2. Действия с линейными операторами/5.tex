\documentclass{article}
\usepackage{mathtext}
\usepackage[russian]{babel}
\usepackage[a3paper, paperwidth=28cm, paperheight=14cm, top=1cm,bottom=1cm,left=1cm,right=1cm,marginparwidth=1.75cm]{geometry}
\usepackage{amsmath}
\usepackage{amssymb}
\usepackage{graphicx}
\usepackage{cancel}
\usepackage{wrapfig}
\pagenumbering{gobble}

\newlength{\tempheight}
\newcommand{\Let}[0]{%
\mathbin{\text{\settoheight{\tempheight}{\mathstrut}\raisebox{0.5\pgflinewidth}{%
\tikz[baseline,line cap=round,line join=round] \draw (0,0) --++ (0.4em,0) --++ (0,1.5ex) --++ (-0.4em,0);%
}}}\;}

\begin{document}
\begin{center}
Решением будет являться $y = Ax$, но такой, что его решение будет состоять из ФСРОС и ЧРНС, т.к. мы ищем $\varphi^{-1}(x)$:
$$\left(\begin{array}{rrrrr|r}
2 & 1 & -3 & -14 & -16 & 26 \\ 1 & 1 & -3 & -13 & -13 & 21 \\ -1 & 0 & 1 & 4 & 6 & -11
\end{array}\right) \sim \left(\begin{array}{rrrrr|r}
0 & 1 & -1 & -6 & -4 & 4 \\ 0 & 1 & -2 & -9 & -7 & 10 \\ -1 & 0 & 1 & 4 & 6 & -11
\end{array}\right) \sim \left(\begin{array}{rrrrr|r}
-1 & 0 & 1 & 4 & 6 & -11 \\ 0 & 1 & -2 & -9 & -7 & 10  \\ 0 & 0 & 1 & 3 & 3 & -6
\end{array}\right)\Leftrightarrow \begin{cases}
x_1 = 11 + x_3 + 4x_4 + 6x_5 \\
x_2 = 10 + 2x_3 + 9x_4 + 7x_5 \\
x_3 = -6 -3x_4 - 3x_5 \\
x_4, x_5 \in \mathbb{R}
\end{cases}$$
\begin{alignat*}{2}
&НС: \begin{cases}
x_1 = 5 + x_4 + 3x_5 \\
x_2 = -2 + 3x_4 + x_5 \\
x_3 = -6 -3x_4 - 3x_5 \\
x_4, x_5 \in \mathbb{R}
\end{cases}& ОС: \begin{cases}
x_1 = x_4 + 3x_5 \\
x_2 = 3x_4 + x_5 \\
x_3 = -3x_4 - 3x_5 \\
x_4, x_5 \in \mathbb{R}
\end{cases} \\ \, \\
&ЧРНС: \left\{\begin{pmatrix}
5 \\ -2 \\ -6 \\ 0 \\ 0
\end{pmatrix}\right\}\qquad\qquad &ФСРОС: \left\{\begin{pmatrix}
1 \\ 3 \\ -3 \\ 1 \\ 0
\end{pmatrix}, \begin{pmatrix}
3 \\ 1 \\ -3 \\ 0 \\ 1
\end{pmatrix}\right\}
\end{alignat*} \\
\end{center}
\end{document}