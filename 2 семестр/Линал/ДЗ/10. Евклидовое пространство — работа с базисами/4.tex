
\documentclass{article}
\usepackage{cmap}
\usepackage{mathtext}
\usepackage[russian]{babel}
\usepackage[a3paper, paperwidth=28cm, paperheight=14cm, top=1cm,bottom=1cm,left=1cm,right=1cm,marginparwidth=1.75cm]{geometry}
\usepackage{amsmath}
\usepackage{amssymb}
\usepackage{multicol}
\usepackage{fancyhdr}
\usepackage{nicefrac}
\usepackage{graphicx}
\usepackage{cancel}
\usepackage{wrapfig}
\usepackage{tikz}
\pagenumbering{gobble}


\newlength{\tempheight}
\newcommand{\Let}[0]{%
\mathbin{\text{\settoheight{\tempheight}{\mathstrut}\raisebox{0.5\pgflinewidth}{%
\tikz[baseline,line cap=round,line join=round] \draw (0,0) --++ (0.4em,0) --++ (0,1.5ex) --++ (-0.4em,0);%
}}}\;}
\newcommand{\e}{\text{e}}
\newcommand{\la}{\lambda}
\newcommand{\shiftleft}[3]{\makebox[#1][r]{\makebox[#2][l]{#3}}}
\newcommand{\shiftright}[3]{\makebox[#2][r]{\makebox[#1][l]{#3}}}
\newcommand*\circled[1]{\tikz[baseline=(char.base)]{
            \node[shape=circle,draw,inner sep=2pt] (char) {#1};}}
\newcommand*\squared[1]{\tikz[baseline=(char.base)]{
            \node[shape=rectangle,draw,inner sep=4pt] (char) {#1};}}
\newcommand{\at}{\biggr\rvert}

\begin{document}
\begin{center}
    Матрица перехода для заданного преобразования базиса будет выглядеть так: $
        T=\begin{pmatrix}
            1 & 0 \\ 0 & -1
        \end{pmatrix}
    $.\\
    Тогда уже не трудно вычислить матрицу Грама в базисе $\left\{\tilde{e}_1, \tilde{e}_2\right\}$:
    $$
        \tilde{G} = T^{T}GT = \begin{pmatrix}
            1 & 0 \\ 0 & -1
        \end{pmatrix}\begin{pmatrix}
            3 & 1 \\ 1 & 1
        \end{pmatrix}\begin{pmatrix}
            1 & 0 \\ 0 & -1
        \end{pmatrix} = \begin{pmatrix}
            3 & 1 \\ -1 & -1
        \end{pmatrix}\begin{pmatrix}
            1 & 0 \\ 0 & -1
        \end{pmatrix} = \begin{pmatrix}
            3 & -1 \\ -1 & 1
        \end{pmatrix}
    $$
    $\tilde{G}$ находится в базисе $\left\{\tilde{e}_1, \tilde{e}_2\right\}$, как и $v_1$ и $v_2$. Теперь мы можем вычислить $\left\langle v_1, v_2\right\rangle $:
    $$\left\langle v_1, v_2\right\rangle = v_1^T\tilde{G}v_2 = \begin{pmatrix}
        -4 & -5
    \end{pmatrix}\begin{pmatrix}
        3 & -1 \\ -1 & 1
    \end{pmatrix}\begin{pmatrix}
        0 \\ -1
    \end{pmatrix} = \begin{pmatrix}
        -7 & -1
    \end{pmatrix}\begin{pmatrix}
        0 \\ -1
    \end{pmatrix} = 1$$
\end{center}
\end{document}