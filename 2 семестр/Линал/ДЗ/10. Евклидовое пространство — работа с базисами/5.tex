
\documentclass{article}
\usepackage{cmap}
\usepackage{mathtext}
\usepackage[russian]{babel}
\usepackage[a3paper, paperwidth=28cm, paperheight=14cm, top=1cm,bottom=1cm,left=1cm,right=1cm,marginparwidth=1.75cm]{geometry}
\usepackage{amsmath}
\usepackage{amssymb}
\usepackage{multicol}
\usepackage{fancyhdr}
\usepackage{nicefrac}
\usepackage{graphicx}
\usepackage{cancel}
\usepackage{wrapfig}
\usepackage{tikz}
\pagenumbering{gobble}


\newlength{\tempheight}
\newcommand{\Let}[0]{%
\mathbin{\text{\settoheight{\tempheight}{\mathstrut}\raisebox{0.5\pgflinewidth}{%
\tikz[baseline,line cap=round,line join=round] \draw (0,0) --++ (0.4em,0) --++ (0,1.5ex) --++ (-0.4em,0);%
}}}\;}
\newcommand{\e}{\text{e}}
\newcommand{\la}{\lambda}
\newcommand{\shiftleft}[3]{\makebox[#1][r]{\makebox[#2][l]{#3}}}
\newcommand{\shiftright}[3]{\makebox[#2][r]{\makebox[#1][l]{#3}}}
\newcommand*\circled[1]{\tikz[baseline=(char.base)]{
            \node[shape=circle,draw,inner sep=2pt] (char) {#1};}}
\newcommand*\squared[1]{\tikz[baseline=(char.base)]{
            \node[shape=rectangle,draw,inner sep=4pt] (char) {#1};}}
\newcommand{\at}{\biggr\rvert}

\begin{document}
\begin{center}
    Найдём матрицу перехода $T$ из базиса $e$ в базис $\tilde{e}$:
    $$
        T = e^{-1}\tilde{e} = \begin{pmatrix}
            1  & -1 & 1  \\
            1  & 0  & 2  \\
            -2 & 1  & -2
        \end{pmatrix}^{-1}\begin{pmatrix}
            -1 & 1 & 1 \\
            -1 & 0 & 0 \\
            0  & 1 & 2
        \end{pmatrix} = \begin{pmatrix}
            -2 & -1 & -2 \\
            -2 & 0  & -1 \\
            1  & 1  & 1
        \end{pmatrix}\begin{pmatrix}
            -1 & 1 & 1 \\
            -1 & 0 & 0 \\
            0  & 1 & 2
        \end{pmatrix} = \begin{pmatrix}
            3  & -4 & -6 \\
            2  & -3 & -4 \\
            -2 & 2  & 3
        \end{pmatrix}
    $$
    Тогда уже не трудно вычислить матрицу Грама в базисе $\left\{\tilde{e}_1, \tilde{e}_2\right\}$:
    $$
        \tilde{G} = T^{T}GT = \begin{pmatrix}
            3  & 2  & -2 \\
            -4 & -3 & 2  \\
            -6 & -4 & 3
        \end{pmatrix}\begin{pmatrix}
            11 & 6 & 2 \\
            6  & 4 & 2 \\
            2  & 2 & 2
        \end{pmatrix}\begin{pmatrix}
            3  & -4 & -6 \\
            2  & -3 & -4 \\
            -2 & 2  & 3
        \end{pmatrix} = \begin{pmatrix}
            41  & 22  & 6   \\
            -58 & -32 & -10 \\
            -84 & -46 & -14
        \end{pmatrix}\begin{pmatrix}
            3  & -4 & -6 \\
            2  & -3 & -4 \\
            -2 & 2  & 3
        \end{pmatrix} = \begin{pmatrix}
            155  & -218 & -316 \\
            -218 & 308  & 446  \\
            -316 & 446  & 646
        \end{pmatrix}
    $$
\end{center}
\end{document}