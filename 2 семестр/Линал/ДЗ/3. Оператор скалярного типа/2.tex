\documentclass{article}
\usepackage{mathtext}
\usepackage[russian]{babel}
\usepackage[a3paper, paperwidth=28cm, paperheight=28cm, top=1cm,bottom=1cm,left=1cm,right=1cm,marginparwidth=1.75cm]{geometry}
\usepackage{amsmath}
\usepackage{amssymb}
\usepackage{graphicx}
\usepackage{cancel}
\usepackage{wrapfig}
\pagenumbering{gobble}
\newcommand{\e}{\text{e}}

\begin{document}
\begin{center}
Найдём собственные значения оператора:
\end{center}
$$\left|\begin{matrix}
-17-\lambda & 24 & 12 & 12 \\
-48 & 55-\lambda & 24 & 24 \\
12 & -12 & 1-\lambda & -6 \\
48 & -48 & -24 & -17-\lambda
\end{matrix}\right| = 0$$
$$\lambda^4-22\lambda^3+168\lambda^2-490\lambda+343 = 0$$
$$(\lambda-1)(\lambda-7)^3=0$$
$$\Downarrow$$
$$\sigma_A = \{1^{(1)},7^{(3)}\}$$ \\
\begin{center}
Найдём собственные вектора:
\end{center}
$$\lambda = 1$$
$$\begin{pmatrix}
-18 & 24 & 12 & 12 \\
-48 & 54 & 24 & 24 \\
12 & -12 & 0 & -6 \\
48 & -48 & -24 & -18
\end{pmatrix} \sim \begin{pmatrix}
3 & -4 & -2 & -2 \\
0 & 5 & 4 & 4 \\
0 & 0 & 4 & -1 \\
0 & 0 & 0 & 0
\end{pmatrix} \Rightarrow \begin{cases}
x_1 = \frac{4x_2}{3} + \frac{2x_3}{3} + \frac{2x_4}{3} \\
x_2 = -\frac{4x_3}{5} - \frac{4x_4}{5} \\
x_3 = \frac{x_4}{4} \\
x_4 \in \mathbb{R}
\end{cases} \Leftrightarrow \begin{cases}
x_1 =- \frac{x_4}{2} \\
x_2 = -x_4\\
x_3 = \frac{x_4}{4} \\
x_4 \in \mathbb{R}
\end{cases} \Rightarrow v_1 = \begin{pmatrix}
-2 \\ -4 \\ 1 \\ 4
\end{pmatrix}$$\\
$$\lambda = 7$$
$$\begin{pmatrix}
-24 & 24 & 12 & 12 \\
-48 & 48 & 24 & 24 \\
12 & -12 & -6 & -6 \\
48 & -48 & -24 & -24
\end{pmatrix} \sim \begin{pmatrix}
-2 & 2 & 1 & 1 \\
0 & 0 & 0 & 0 \\
0 & 0 & 0 & 0 \\
0 & 0 & 0 & 0
\end{pmatrix} \Rightarrow \begin{cases}
x_1 = x_2 + \frac{x_3}{2} + \frac{x_4}{2} \\
x_2 \in \mathbb{R} \\
x_3 \in \mathbb{R} \\
x_4 \in \mathbb{R} \\
\end{cases} \Rightarrow v_2 = \begin{pmatrix}
1 \\ 1 \\ 0 \\ 0
\end{pmatrix} \quad v_3 = \begin{pmatrix}
1 \\ 0 \\ 2 \\ 0
\end{pmatrix} \quad v_4 = \begin{pmatrix}
1 \\ 0 \\ 0 \\ 2
\end{pmatrix}$$ \\

\begin{center}
Оператор проектирования $P_{\lambda=7} = P_{v_2} + P_{v_3} + P_{v_4}$, где каждый из $P_{v_n} = u^n(e_1|e_2|e_3|e_4)v_n$, в котором $u^n$ --- n-ая строка сопряжённого к базису $\{v_1, v_2, v_3, v_4\}$.
\end{center}
$$u = \begin{pmatrix}
-2 & 2 & 1 & 1 \\
-8 & 9 & 4 & 4 \\
1 & -1 & 0 & -0.5 \\
4 & -4 & -2 & -1.5
\end{pmatrix}$$
$$\Downarrow$$
$$P_{v_2} = \begin{pmatrix}
| & | & | & |\\
u^2e_1v_2 & u^2e_2v_2 & u^2e_3v_2 & u^2e_4v_2 \\
| & | & | & |
\end{pmatrix} = \begin{pmatrix}
-8 & 9 & 4 & 4 \\
-8 & 9 & 4 & 4 \\
0 & 0 & 0 & 0 \\
0 & 0 & 0 & 0
\end{pmatrix}$$ \\
$$P_{v_3} = \begin{pmatrix}
| & | & | & |\\
u^3e_1v_3 & u^3e_2v_3 & u^3e_3v_3 & u^3e_4v_3 \\
| & | & | & |
\end{pmatrix} = \begin{pmatrix}
1 & -1 & 0 & -0.5 \\
0 & 0 & 0 & 0 \\
2 & -2 & 0 & -1 \\
0 & 0 & 0 & 0
\end{pmatrix} \qquad\quad P_{v_4} = \begin{pmatrix}
| & | & | & |\\
u^4e_1v_4 & u^4e_2v_4 & u^4e_3v_4 & u^4e_4v_4 \\
| & | & | & |
\end{pmatrix} = \begin{pmatrix}
4 & -4 & -2 & -1.5 \\
0 & 0 &0 & 0 \\
0 & 0 &0 & 0 \\
8 & -8 & -4 & -3
\end{pmatrix}$$\\
$$P_{\lambda=7} = P_{v_2} + P_{v_3} + P_{v_4} = \begin{pmatrix}
-8 & 9 & 4 & 4 \\
-8 & 9 & 4 & 4 \\
0 & 0 & 0 & 0 \\
0 & 0 & 0 & 0
\end{pmatrix} +  \begin{pmatrix}
1 & -1 & 0 & -0.5 \\
0 & 0 & 0 & 0 \\
2 & -2 & 0 & -1 \\
0 & 0 & 0 & 0
\end{pmatrix} + \begin{pmatrix}
4 & -4 & -2 & -1.5 \\
0 & 0 &0 & 0 \\
0 & 0 &0 & 0 \\
8 & -8 & -4 & -3
\end{pmatrix} = \begin{pmatrix}
-3 & 4 & 2 & 2\\
-8 & 9 & 4 &  4 \\
2 & -2 & 0 & -1 \\
8 & -8 & -4 & -3
\end{pmatrix}$$
\end{document}