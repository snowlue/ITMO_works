\documentclass{article}
\usepackage{mathtext}
\usepackage[russian]{babel}
\usepackage[a3paper, paperwidth=28cm, paperheight=35cm, top=1cm,bottom=1cm,left=1cm,right=1cm,marginparwidth=1.75cm]{geometry}
\usepackage{amsmath}
\usepackage{amssymb}
\usepackage{multicol}
\usepackage{fancyhdr}
\usepackage{nicefrac}
\usepackage{graphicx}
\usepackage{cancel}
\usepackage{wrapfig}
\pagenumbering{gobble}


\newlength{\tempheight}
\newcommand{\Let}[0]{%
\mathbin{\text{\settoheight{\tempheight}{\mathstrut}\raisebox{0.5\pgflinewidth}{%
\tikz[baseline,line cap=round,line join=round] \draw (0,0) --++ (0.4em,0) --++ (0,1.5ex) --++ (-0.4em,0);%
}}}\;}
\newcommand{\e}{\text{e}}
\newcommand{\la}{\lambda}
\newcommand{\shiftleft}[3]{\makebox[#1][r]{\makebox[#2][l]{#3}}}
\newcommand{\shiftright}[3]{\makebox[#2][r]{\makebox[#1][l]{#3}}}

\begin{document}
\begin{center}
Найдём характеристический полином и, как следствие, спектр оператора:
\end{center}
$$\left|\begin{matrix}
19-\la & 8 & 4 & 2 & 1 \\
-1 & 3-\la & 0 & 0 & 0 \\
-27 & -14 & -4-\la & -3 & -2 \\
-31 & -16 & -8 & -2-\la & -1 \\
-47 & -24 & -12 & -6 & -1-\la
\end{matrix}\right| = -\la^5+15\la^4-88\la^3+252\la^2-352\la+192 = -(\la-4)^{2}(\la-3)(\la-2)^{2} \Rightarrow \sigma_\varphi = \left\{2^{(2)}, 4^{(2)}, 3^{(1)}\right\}$$ \\
\begin{center}
Найдём к собственным значениям оператора собственные вектора:
\end{center}
$$\la = 2$$
$$\left(\begin{matrix}
17 & 8 & 4 & 2 & 1 \\
-1 & 1 & 0 & 0 & 0 \\
-27 & -14 & -6 & -3 & -2 \\
-31 & -16 & -8 & -4 & -1 \\
-47 & -24 & -12 & -6 & -3
\end{matrix}\right) \sim \left(\begin{matrix}
17 & 8 & 4 & 2 & 1 \\
-1 & 1 & 0 & 0 & 0 \\
10 & 6 & 2 & 1 & 1 \\
3 & 0 & 0 & 0 & 1 \\
4 & 0 & 0 & 0 & 0
\end{matrix}\right) \sim \left(\begin{matrix}
1 & 0 & 0 & 0 & 0 \\
0 & 8 & 4 & 2 & 1 \\
0 & 1 & 0 & 0 & 0 \\
0 & 0 & 2 & 1 & 1 \\
0 & 0 & 0 & 0 & 1 \\
\end{matrix}\right) \sim \left(\begin{matrix}
1 & 0 & 0 & 0 & 0 \\
0 & 1 & 0 & 0 & 0 \\
0 & 0 & 2 & 1 & 1 \\
0 & 0 & 0 & 0 & -1 \\
0 & 0 & 0 & 0 & 1 \\
\end{matrix}\right) \Rightarrow \begin{cases}
\xi_1 = 0 \\ \xi_2 = 0 \\ \xi_3 = -\nicefrac{\xi_4}{2} \\ \xi_4 \in \mathbb{R} \\ \xi_5 = 0
\end{cases} \Rightarrow v_1 = \begin{pmatrix}
0 \\ 0 \\ -1 \\ 2 \\ 0
\end{pmatrix}$$
$$\la = 4$$
$$\left(\begin{matrix}
15 & 8 & 4 & 2 & 1 \\
-1 & -1 & 0 & 0 & 0 \\
-27 & -14 & -8 & -3 & -2 \\
-31 & -16 & -8 & -6 & -1 \\
-47 & -24 & -12 & -6 & -5 \\
\end{matrix}\right) \sim \left(\begin{matrix}
15 & 8 & 4 & 2 & 1 \\
-1 & -1 & 0 & 0 & 0 \\
3 & 2 & 0 & 1 & 0 \\
-1 & 0 & 0 & -2 & 1 \\
-2 & 0 & 0 & 0 & -2 \\
\end{matrix}\right) \sim \left(\begin{matrix}
0 & 0 & 4 & -5 & 1 \\
1 & 1 & 0 & 0 & 0 \\
1 & 0 & 0 & 1 & 0 \\
-2 & 0 & 0 & -2 & 0 \\
1 & 0 & 0 & 0 & 1 \\
\end{matrix}\right) \sim \left(\begin{matrix}
1 & 1 & 0 & 0 & 0 \\
0 & 1 & 0 & -1 & 0 \\
0 & 0 & 4 & -5 & 1 \\
0 & 0 & 0 & -1 & 1 \\
0 & 0 & 0 & 0 & 0 \\
\end{matrix}\right) \Rightarrow \begin{cases}
\xi_1 = -\xi_5 \\
\xi_2 = \xi_5 \\
\xi_3 = \xi_5 \\
\xi_4 = \xi_5 \\
\xi_5 \in \mathbb{R}
\end{cases} \Rightarrow v_2 = \begin{pmatrix}
-1 \\ 1 \\ 1 \\ 1 \\ 1
\end{pmatrix}$$

$$\la = 3$$
$$\left(\begin{matrix}
16 & 8 & 4 & 2 & 1 \\
-1 & 0 & 0 & 0 & 0 \\
-27 & -14 & -7 & -3 & -2 \\
-31 & -16 & -8 & -5 & -1 \\
-47 & -24 & -12 & -6 & -4 \\
\end{matrix}\right) \sim \left(\begin{matrix}
1 & 0 & 0 & 0 & 0 \\
16 & 8 & 4 & 2 & 1 \\
5 & 2 & 1 & 1 & 0 \\
1 & 0 & 0 & -1 & 1 \\
1 & 0 & 0 & 0 & -1 \\
\end{matrix}\right) \sim \left(\begin{matrix}
1 & 0 & 0 & 0 & 0 \\
0 & 2 & 1 & -1 & 1 \\
0 & 2 & 1 & 1 & 0 \\
0 & 0 & 0 & -1 & 1 \\
0 & 0 & 0 & 0 & -1 \\
\end{matrix}\right) \sim \left(\begin{matrix}
1 & 0 & 0 & 0 & 0 \\
0 & 2 & 1 & 0 & 0 \\
0 & 0 & 0 & 0 & 0 \\
0 & 0 & 0 & 1 & 0 \\
0 & 0 & 0 & 0 & 1 \\
\end{matrix}\right) \Rightarrow \begin{cases}
\xi_1 = 0 \\ \xi_2 = -\frac{\xi_3}{2} \\ \xi_3 \in \mathbb{R} \\ \xi_4 = 0 \\ \xi_5 = 0
\end{cases} \Rightarrow v_3 = \begin{pmatrix}
0 \\ -1 \\ 2 \\ 0 \\ 0
\end{pmatrix}$$ \\
\begin{center}
Собственных векторов недостаточно для базиса. Перед нами оператор нескалярного типа. \\
Найдём к $v_1$ и $v_2$ ещё два присоединённых вектора, чтобы дополнить собственные до базиса.
\end{center}
$$\left(\begin{array}{ccccc|c}
17 & 8 & 4 & 2 & 1 & 0\\
-1 & 1 & 0 & 0 & 0 & 0\\
-27 & -14 & -6 & -3 & -2 & -1 \\
-31 & -16 & -8 & -4 & -1 & 2\\
-47 & -24 & -12 & -6 & -3 & 0
\end{array}\right) \sim \left(\begin{array}{ccccc|c}
17 & 8 & 4 & 2 & 1 & 0\\
-1 & 1 & 0 & 0 & 0 & 0\\
10 & 6 & 2 & 1 & 1 & 1 \\
3 & 0 & 0 & 0 & 1 & 2\\
4 & 0 & 0 & 0 & 0 & 0
\end{array}\right) \sim \left(\begin{array}{ccccc|c}
1 & 0 & 0 & 0 & 0 & 0\\
0 & 1 & 0 & 0 & 0 & 0\\
0 & 0 & 2 & 1 & 0 & -1\\
0 & 0 & 0 & 0 & 1 & 2 \\
0 & 0 & 0 & 0 & 1 & 2\\
\end{array}\right) \Rightarrow \begin{cases}
\xi_1 = 0 \\ \xi_2 = 0 \\ \xi_3 = -\nicefrac{\xi_4}{2} - 0.5 \\ \xi_4 \in \mathbb{R} \\ \xi_5 = 2
\end{cases} \Rightarrow u_1 = \begin{pmatrix}
0 \\ 0 \\ -1 \\ 1 \\ 2
\end{pmatrix}$$
$$\left(\begin{array}{ccccc|c}
15 & 8 & 4 & 2 & 1 & -1\\
-1 & -1 & 0 & 0 & 0 & 1\\
-27 & -14 & -8 & -3 & -2 & 1 \\
-31 & -16 & -8 & -6 & -1 & 1\\
-47 & -24 & -12 & -6 & -5 & 1\\
\end{array}\right) \sim \left(\begin{array}{ccccc|c}
15 & 8 & 4 & 2 & 1 & -1\\
1 & 1 & 0 & 0 & 0 & -1\\
3 & 2 & 0 & 1 & 0 & -1 \\
-1 & 0 & 0 & -2 & 1 & -1\\
-2 & 0 & 0 & 0 & -2 & -2\\
\end{array}\right) \sim \left(\begin{array}{ccccc|c}
1 & 0 & 0 & 0 & 1 & 1 \\
0 & 1 & 0 & 0 & -1 & -2\\
0 & 0 & 4 & 0 & -4 & 0\\
0 & 0 & 0 & -1 & 1 & 0\\
0 & 0 & 0 & 2 & -2 & 0\\
\end{array}\right) \Rightarrow \begin{cases}
\xi_1 = -\xi_5 + 1 \\
\xi_2 = \xi_5 - 2 \\
\xi_3 = \xi_5 \\
\xi_4 = \xi_5 \\
\xi_5 \in \mathbb{R}
\end{cases} \Rightarrow u_2 = \begin{pmatrix}
0 \\ -1 \\ 1 \\ 1 \\ 1
\end{pmatrix}$$ \\
\begin{center}
В базисе $T=\left\{v_1, u_1, v_2, u_2, v_3\right\}$ жорданова форма оператора будет выглядеть так:
\end{center}
$$\mathcal{J}_\varphi = \begin{pmatrix}
2 & 1 & 0 & 0 & 0 \\
0 & 2 & 0 & 0 & 0 \\
0 & 0 & 4 & 1 & 0 \\
0 & 0 & 0 & 4 & 0 \\
0 & 0 & 0 & 0 & 3 \\
\end{pmatrix}$$
\begin{center}
Тогда значение функции от оператора $f(\varphi) = \e^{0.5\varphi}$ вычисляется следующим образом:
\end{center}
$$f(\varphi) = Tf(\mathcal{J_\varphi})T^{-1} = \begin{pmatrix}
0 & 0 & -1 & 0 & 0 \\
0 & 0 & 1 & -1 & -1 \\
-1 & -1 & 1 & 1 & 2 \\
2 & 1 & 1 & 1 & 0 \\
0 & 2 & 1 & 1 & 0 \\
\end{pmatrix}\times\begin{pmatrix}
\e & 0.5\e & 0 & 0 & 0 \\
0 & \e & 0 & 0 & 0 \\
0 & 0 & \e^2 & 0.5\e^2 & 0 \\
0 & 0 & 0 & \e^2 & 0 \\
0 & 0 & 0 & 0 & \e^{1.5} \\
\end{pmatrix}\times\begin{pmatrix}
4 & 2 & 1 & 1 & 0 \\
8 & 4 & 2 & 1 & 1 \\
-1 & 0 & 0 & 0 & 0 \\
-15 & -8 & -4 & -2 & -1 \\
14 & 7 & 4 & 2 & 1
\end{pmatrix} = \begin{pmatrix}
0 & 0 & -\e^2 & -0.5\e^2 & 0 \\
0 & 0 & \e^2 & -0.5\e^2 & -\e^{1.5} \\
-\e & -1.5\e & \e^2 & 1.5\e^2 & 2\e^{1.5} \\
2\e & 2\e & \e^2 & 1.5\e^2 & 0 \\
0 & 2\e & \e^2 & 1.5\e^2 & 0
\end{pmatrix}\times$$
$$\times \begin{pmatrix}
4 & 2 & 1 & 1 & 0 \\
8 & 4 & 2 & 1 & 1 \\
-1 & 0 & 0 & 0 & 0 \\
-15 & -8 & -4 & -2 & -1 \\
14 & 7 & 4 & 2 & 1
\end{pmatrix} = \left(\begin{matrix}
8.5\e^2 & 4\e^2 & 2\e^2 & \e^2 & 0.5\e^2 \\
6.5\e^2-14\e^{1.5} & 4\e^2-7\e^{1.5} & 2(\e^2-2\e^{1.5}) & \e^2-2\e^{1.5} & 0.5\e^2-\e^{1.5} \\
28\e^{1.5}-23.5\e^2-16\e & 2(7\e^{1.5}-6\e^2-4\e) & 2(4\e^{1.5}-3\e^2-2\e) & 4\e^{1.5}-3\e^2-2.5\e & 2\e^{1.5}-1.5\e^2-1.5\e \\
24\e-23.5\e^2 & 12(\e-\e^2) & 6(\e-\e^2) & 4\e-3\e^2 & 2\e-1.5\e^2 \\
16\e-23.5\e^2 & 4(2\e-3\e^2) & 2(2\e-3\e^2) & 2\e-3\e^2 & 2\e-1.5\e^2
\end{matrix}\right) \approx$$
$$\approx \begin{pmatrix}
62.81 & 29.56 & 14.78 & 7.39 & 3.69 \\
-14.71 & -1.82 & -3.15 & -1.57 & -0.79 \\
-91.65 & -47.67 & -19.35 & -11.04 & -6.20 \\
-108.40 & -56.05 & -28.02 & -11.29 & -5.65 \\
-130.15 & -66.92 & -33.46 & -16.73 & -5.65
\end{pmatrix}$$
\end{document}