\documentclass{article}
\usepackage{mathtext}
\usepackage[russian]{babel}
\usepackage[a3paper, paperwidth=30cm, paperheight=20cm, top=1cm,bottom=1cm,left=1cm,right=1cm,marginparwidth=1.75cm]{geometry}
\usepackage{amsmath}
\usepackage{amssymb}
\usepackage{multicol}
\usepackage{fancyhdr}
\usepackage{nicefrac}
\usepackage{graphicx}
\usepackage{cancel}
\usepackage{wrapfig}
\usepackage{tikz}
\pagenumbering{gobble}


\newlength{\tempheight}
\newcommand{\Let}[0]{%
\mathbin{\text{\settoheight{\tempheight}{\mathstrut}\raisebox{0.5\pgflinewidth}{%
\tikz[baseline,line cap=round,line join=round] \draw (0,0) --++ (0.4em,0) --++ (0,1.5ex) --++ (-0.4em,0);%
}}}\;}
\newcommand{\e}{\text{e}}
\newcommand{\la}{\lambda}
\newcommand{\shiftleft}[3]{\makebox[#1][r]{\makebox[#2][l]{#3}}}
\newcommand{\shiftright}[3]{\makebox[#2][r]{\makebox[#1][l]{#3}}}
\newcommand*\circled[1]{\tikz[baseline=(char.base)]{
            \node[shape=circle,draw,inner sep=2pt] (char) {#1};}}
\newcommand*\squared[1]{\tikz[baseline=(char.base)]{
            \node[shape=rectangle,draw,inner sep=4pt] (char) {#1};}}
\newcommand{\at}{\biggr\rvert}

\begin{document}
\begin{center}
Вычислим характеристический полином и, как следствие, найдём спектр оператора:
$$\chi_\varphi(\la) = \begin{vmatrix}
4-\la & 1 & 0 & 0 & 0 \\
13 & 9-\la & 4 & 2 & 1 \\
81 & 40 & 23-\la & 14 & 7 \\
-143 & -72 & -36 & -23-\la & -9 \\
-143 & -72 & -36 & -18 & -14-\la \\
\end{vmatrix} = -\la^5-\la^4+38\la^3-18\la^2-405\la+675 = -(\la - 3)^3(\la + 5)^2 \Rightarrow \sigma_\varphi = \left\{3^{(3)}, -5^{(2)}\right\}$$
Найдём собственные вектора оператора к соответствующим собственным значениям:
$$\la = 3$$
$$\begin{pmatrix}
1 & 1 & 0 & 0 & 0 \\
13 & 6 & 4 & 2 & 1 \\
81 & 40 & 20 & 14 & 7 \\
-143 & -72 & -36 & -26 & -9 \\
-143 & -72 & -36 & -18 & -17
\end{pmatrix} \sim \begin{pmatrix}
1 & 1 & 0 & 0 & 0 \\
0 & -7 & 4 & 2 & 1 \\
0 & -41 & 20 & 14 & 7 \\
0 & -71 & 36 & 26 & 9 \\
0 & 0 & 0 & 8 & -8
\end{pmatrix} \sim \begin{pmatrix}
1 & 1 & 0 & 0 & 0 \\
0 & -7 & 4 & 3 & 0 \\
0 & 1 & -4 & 3 & 0 \\
0 & -1 & -4 & 5 & 0 \\
0 & 0 & 0 & 1 & -1
\end{pmatrix} \sim \begin{pmatrix}
1 & 1 & 0 & 0 & 0 \\
0 & 8 & -8 & 0 & 0 \\
0 & 1 & -4 & 3 & 0 \\
0 & 0 & -8 & 8 & 0 \\
0 & 0 & 0 & 1 & -1
\end{pmatrix} \sim \begin{pmatrix}
1 & 1 & 0 & 0 & 0 \\
0 & 1 & -1 & 0 & 0 \\
0 & 0 & -3 & 3 & 0 \\
0 & 0 & -1 & 1 & 0 \\
0 & 0 & 0 & 1 & -1
\end{pmatrix} \Rightarrow \begin{cases}
\xi_1 = -\xi_2 \\
\xi_2 = \xi_3 \\
\xi_3 = \xi_4 \\
\xi_4 = \xi_5 \\
\xi_5 \in \mathbb{R}
\end{cases} \shiftleft{-2pt}{14pt}{$\Rightarrow$} v_1 = \begin{pmatrix}
-1 \\ 1 \\ 1 \\ 1 \\ 1
\end{pmatrix}$$
$$\la = -5$$
$$\begin{pmatrix}
9 & 1 & 0 & 0 & 0 \\
13 & 14 & 4 & 2 & 1 \\
81 & 40 & 28 & 14 & 7 \\
-143 & -72 & -36 & -18 & -9 \\
-143 & -72 & -36 & -18 & -9 \\
\end{pmatrix} \sim \begin{pmatrix}
9 & 1 & 0 & 0 & 0 \\
4 & 13 & 4 & 2 & 1 \\
0 & 31 & 28 & 14 & 7 \\
0 & 82 & 8 & 4 & 2 \\
0 & 0 & 0 & 0 & 0 \\
\end{pmatrix} \sim \begin{pmatrix}
9 & 1 & 0 & 0 & 0 \\
4 & 13 & 4 & 2 & 1 \\
0 & 31 & 28 & 14 & 7 \\
-4 & 28 & 0 & 0 & 0 \\
0 & 0 & 0 & 0 & 0 \\
\end{pmatrix} \sim \begin{pmatrix}
0 & 64 & 0 & 0 & 0 \\
0 & 41 & 4 & 2 & 1 \\
0 & 31 & 28 & 14 & 7 \\
-1 & 7 & 0 & 0 & 0 \\
0 & 0 & 0 & 0 & 0 \\
\end{pmatrix} \sim \begin{pmatrix}
0 & 1 & 0 & 0 & 0 \\
0 & 0 & 4 & 2 & 1 \\
0 & 0 & 28 & 14 & 7 \\
1 & 0 & 0 & 0 & 0 \\
0 & 0 & 0 & 0 & 0 \\
\end{pmatrix} \Rightarrow \begin{cases}
\xi_1 = 0 \\
\xi_2 = 0 \\
\xi_3, \xi_4 \in \mathbb{R} \\
\xi_5 = -4\xi_3-2\xi_4 \\
\end{cases}  \shiftleft{-15pt}{14pt}{$\Rightarrow$} v_2=\begin{pmatrix}
0 \\ 0 \\ 1 \\ 0 \\ -4
\end{pmatrix}\quad v_3 = \begin{pmatrix}
0 \\ 0 \\ 0 \\ 1 \\ -2
\end{pmatrix}$$
Перед нами оператор нескалярного типа, базис которого будет составлен из 3 собственных векторов и 2 присоединённых и будет выглядеть так: $\left\{v_1, u_1, u_2, v_2, v_3\right\}$, \\ где $u_1$ и $u_2$ --- присоединённые вектора. Тогда жорданова форма оператора в этом базиса примет следующий вид:
$$\mathcal{J} = \begin{pmatrix}
3 & 1 & 0 & 0 & 0 \\
0 & 3 & 1 & 0 & 0 \\
0 & 0 & 3 & 0 & 0 \\
0 & 0 & 0 & -5 & 0 \\
0 & 0 & 0 & 0 & -5 \\
\end{pmatrix}$$
\end{center}
\end{document}