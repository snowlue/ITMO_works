\documentclass[a4paper]{article}
\usepackage{cmap}
\usepackage[utf8]{inputenc}
\usepackage[T2A]{fontenc}
\usepackage[english,russian]{babel} 
\usepackage[left=10mm, top=10mm, right=10mm, bottom=20mm, nohead, nofoot]{geometry}
\usepackage{blindtext}  % рыба-текст
\usepackage{graphicx}  % изобржаения
\usepackage{float} % плавающие объекты
\usepackage{wrapfig}  % изобржаения
\usepackage{tikz} % графика
\usepackage{mdframed} % рамки
\usepackage{xcolor} % определение цветов
\usepackage{nicefrac} % красивые дроби
\usepackage{cancel} % сокращение
\usepackage{multicol}
\usepackage{amsmath,amsfonts,amssymb} % математический пакет
\usepackage{hyperref}  % гиперссылки
\usepackage{fancybox,fancyhdr} % хедер и футер
\usepackage{listings} % код
\pagestyle{fancy}
\fancyhf{}
\fancyhead[L]{Задачи №1494, 1067, 1322}
\fancyhead[R]{Овчинников П.А., АиСД 3.3}
\fancyfoot[C]{\thepage}
\headsep=8mm
\footskip=8mm
\setlength{\parindent}{0mm}

\definecolor{urlcolor}{HTML}{3454D1}
\definecolor{linkcolor}{HTML}{3454D1}
\hypersetup{pdfstartview=FitH, linkcolor=linkcolor, urlcolor=urlcolor, colorlinks=true}

\definecolor{strings}{rgb}{0,0.6,0}
\definecolor{comments}{rgb}{0,0.3,0}
\definecolor{numbers}{rgb}{0.5,0.5,0.5}
\definecolor{keywords}{rgb}{0.09,0.61,0.95}
\definecolor{background}{rgb}{0.97,0.97,0.97}
\lstdefinestyle{codestyle}{
    backgroundcolor=\color{background},
    commentstyle=\color{comments},
    keywordstyle=\color{keywords},
    stringstyle=\color{strings},
    numberstyle=\tiny\color{numbers},
    basicstyle=\ttfamily\footnotesize,
    breakatwhitespace=false,
    breaklines=true,
    captionpos=b,
    inputencoding=utf8,
    keepspaces=true,
    numbers=left,
    numbersep=5pt,
    showspaces=false,
    showstringspaces=false,
    showtabs=false,
    tabsize=2,
    extendedchars=true,
    literate=
    {а}{{\cyra}}1
    {б}{{\cyrb}}1
    {в}{{\cyrv}}1
    {г}{{\cyrg}}1
    {д}{{\cyrd}}1
    {е}{{\cyre}}1
    {ж}{{\cyrzh}}1
    {з}{{\cyrz}}1
    {и}{{\cyri}}1
    {й}{{\cyrishrt}}1
    {к}{{\cyrk}}1
    {л}{{\cyrl}}1
    {м}{{\cyrm}}1
    {н}{{\cyrn}}1
    {о}{{\cyro}}1
    {п}{{\cyrp}}1
    {р}{{\cyrr}}1
    {с}{{\cyrs}}1
    {т}{{\cyrt}}1
    {у}{{\cyru}}1
    {ф}{{\cyrf}}1
    {х}{{\cyrh}}1
    {ц}{{\cyrc}}1
    {ч}{{\cyrch}}1
    {ш}{{\cyrsh}}1
    {щ}{{\cyrshch}}1
    {ъ}{{\cyrhrdsn}}1
    {ы}{{\cyrery}}1
    {ь}{{\cyrsftsn}}1
    {э}{{\cyrerev}}1
    {ю}{{\cyryu}}1
    {я}{{\cyrya}}1
    {А}{{\CYRA}}1
    {Б}{{\CYRB}}1
    {В}{{\CYRV}}1
    {Г}{{\CYRG}}1
    {Д}{{\CYR96}}1
    {Е}{{\CYRE}}1
    {Ж}{{\CYRZH}}1
    {З}{{\CYRZ}}1
    {И}{{\CYRI}}1
    {Й}{{\CYRISHRT}}1
    {К}{{\CYRK}}1
    {Л}{{\CYRL}}1
    {М}{{\CYRM}}1
    {Н}{{\CYRN}}1
    {О}{{\CYRO}}1
    {П}{{\CYRP}}1
    {Р}{{\CYRR}}1
    {С}{{\CYRS}}1
    {Т}{{\CYRT}}1
    {У}{{\CYRU}}1
    {Ф}{{\CYRF}}1
    {Х}{{\CYRH}}1
    {Ц}{{\CYRC}}1
    {Ч}{{\CYRCH}}1
    {Ш}{{\CYRSH}}1
    {Щ}{{\CYRSHCH}}1
    {Ъ}{{\CYRHRDSN}}1
    {Ы}{{\CYRERY}}1
    {Ь}{{\CYRSFTSN}}1
    {Э}{{\CYREREV}}1
    {Ю}{{\CYRYU}}1
    {Я}{{\CYRYA}}1
}

\lstset{style=codestyle}

\addto\captionsrussian{
  \renewcommand{\contentsname}
    {\centering Содержание}
}
\newcommand{\addsection}[1]{
    \phantomsection
    \addcontentsline{toc}{section}{#1}
    \section*{\centering #1}
}
\newcommand{\addsubsection}[1]{
    \phantomsection
    \addcontentsline{toc}{subsection}{#1}
    \subsection*{\centering #1}
}
\newcommand{\addsubsubsection}[1]{
    \phantomsection
    \addcontentsline{toc}{subsubsection}{#1}
    \subsubsection*{\centering #1}
}

\newmdenv[
  leftmargin = 0.5em,
  skipabove = 0.5em,
  skipbelow = 0.5em,
  linewidth = 1pt,
  rightline = false,
  topline = false,
  bottomline = false
]{quotebox}

\newlength{\tempheight}
\newcommand{\Let}{
\mathbin{\text{\settoheight{\tempheight}{\mathstrut}\raisebox{0.4\pgflinewidth}{
\tikz[baseline=0.5ex,line cap=round,line join=round] \draw (0,0) --++ (0.3em,0) --++ (0,2.3ex) --++ (-0.3em,0);
}}}}
\newcommand*\squared[1]{\tikz[baseline=(char.base)]{
            \node[shape=rectangle,draw,inner sep=4pt] (char) {#1};}}
\newcommand*\msquared[1]{\tikz[baseline=(char.base)]{
            \node[shape=rectangle,draw,inner sep=4pt] (char) {$\displaystyle #1$};}}
\newcommand{\at}{\biggr\rvert}
\newcommand{\shiftright}[3]{\makebox[#2][r]{\makebox[#1][l]{#3}}}
\newcommand{\e}{\;\text{e}}
\let\oldint\int
\def\int{\oldint\limits}
\DeclareRobustCommand{\divby}{%
  \mathrel{\vbox{\baselineskip.65ex\lineskiplimit0pt\hbox{.}\hbox{.}\hbox{.}}}%
}

\newcommand\NB{\textbf{N\kern-0.32em\textcolor{red}{B}}}

\begin{document}
\addsection{\href{https://acm.timus.ru/problem.aspx?space=1&num=1494}{Задача №1494. Монобильярд}}
Стол для монобильярда, установленный в игровом доме уездного города $N$, оказался очень прибыльным вложением. До того, как в городе появился небезызвестный господин Чичиков. Раз за разом он выигрывал, и хозяин, подсчитывая убытки, понимал, что дело тут нечисто. Однако уличить подлеца в жульничестве не удавалось до прибытия в город $N$ ревизора из Петербурга.\\[0.5em]
Правила игры в монобильярд очень просты: нужно последовательно закатить в единственную лузу шары с номерами $1, 2, \dots, N$ (именно в этом порядке). Пока господин Чичиков играл, ревизор несколько раз подходил к столу и забирал из лузы последний закатившийся туда шар. В конце концов, оказалось, что Чичиков закатил в лузу все шары, а ревизор все шары достал и обследовал. Аферист утверждал, что закатил шары в правильном порядке. Хозяин понял, что это его шанс: ревизор должен помнить, в каком порядке он доставал шары. Однако так ли легко будет доказать жульничество?\\[1em]
\textbf{Исходные данные:}
\begin{quotebox}
    В первой строке записано целое число $N$ — количество бильярдных шаров $(1 \leq N \leq 10^5)$. В следующих $N$ строках даны номера этих шаров в том порядке, в котором ревизор забирал их из лузы.
\end{quotebox}
\textbf{Результат:}
\begin{quotebox}
    Выведите слово «Cheater», если Чичиков не мог закатить все $N$ шаров в правильном порядке. Иначе выведите «Not a proof».
\end{quotebox}
\addsubsection{Рабочий код}
\begin{lstlisting}[language=c++]
#include <iostream>
#include <stack>
using namespace std;
int main() {
    int n;
    int numberBall;
    int lastBallStack = 1;
    bool cheaterStatus = false;
    stack <int> inBall;
    cin >> n;
    cin >> numberBall;
    for (int j=lastBallStack; j<numberBall; ++j) { inBall.push(j); }
    lastBallStack = numberBall;
    for (int i=1; i<n; i++) {
        std::cin >> numberBall;
        if (inBall.empty() || numberBall > inBall.top()) {
            for (int j=lastBallStack+1; j<numberBall; ++j) { inBall.push(j); }
            lastBallStack = numberBall;
        }
        else if (numberBall == inBall.top()){ inBall.pop(); }
        else { cheaterStatus = true; }
    }
    if (cheaterStatus) { cout << "Cheater"; }
    else { cout << "Not a proof"; }
}
\end{lstlisting}
\addsubsection{Объяснение алгоритма}
Алгоритм предполагает использования стека для отслеживания ожидаемых чисел. Если следующее число больше вершины стека, оно добавляет недостающие числа в стек. Если число совпадает с вершиной стека, оно удаляется. Если число меньше и не совпадает с вершиной, алгоритм определяет, что было совершено «читерство». И в конце, если обнаружено «читерство», выводится «Cheater», иначе --- «Not a proof».

\newpage
\addsection{\href{https://acm.timus.ru/problem.aspx?space=1&num=1067}{Задача №1067. Структура папок}}
Хакер Билл случайно потерял всю информацию с жесткого диска своего компьютера, и у него нет резервных копий его содержимого. Но он сожалеет не о потере самих файлов, а о потере очень понятной и удобной структуры папок, которую он создавал и сохранял в течение многих лет работы.\\[0.5em]
К счастью, у Билла есть несколько копий списков папок с его жесткого диска. С помощью этих списков он смог восстановить полные пути к некоторым папкам (например, «WINNT\textbackslash SYSTEM32\textbackslash CERTSRV\textbackslash CERTCO~1\textbackslash X86»). Он поместил их все в файл, записав каждый найденный путь в отдельную строку.\\[0.5em]
Напишите программу, которая восстановит структуру папок Билла и выведет ее в виде отформатированного дерева.\\[1em]
\textbf{Исходные данные:}
\begin{quotebox}
    Первая строка содержит целое число $N$ – количество различных путей к папкам $(1 \leq N \leq 500)$. Далее следуют N строк с путями к папкам. Каждый путь занимает одну строку и не содержит пробелов, в том числе, начальных и конечных. Длина каждого пути не превышает 80 символов. Каждый путь встречается в списке один раз и состоит из нескольких имен папок, разделенных обратной косой чертой («\textbackslash»).\\[0.5em]
    Имя каждой папки состоит из 1-8 заглавных букв, цифр или специальных символов из следующего списка: восклицательный знак, решетка, знак доллара, знак процента, амперсанд, апостроф, открывающаяся и закрывающаяся скобки, знак дефиса, собаки, циркумфлекс, подчеркивание, гравис, открывающаяся и закрывающаяся фигурная скобка и тильда \verb|(«!\#\$\%\&'()-@^_`\{\}~»)|.
\end{quotebox}
\textbf{Результат:}
\begin{quotebox}
    Выведите отформатированное дерево папок. Каждое имя папки должно быть выведено в отдельной строке, перед ним должно стоять несколько пробелов, указывающих на глубину этой папки в иерархии. Подпапки должны быть перечислены в лексикографическом порядке непосредственно после их родительской папки; перед их именем должно стоять на один пробел больше, чем перед именем их родительской папки. Папки верхнего уровня выводятся без пробелов и также должны быть перечислены в лексикографическом порядке.
\end{quotebox}
\addsubsection{Рабочий код}
\begin{lstlisting}[language=c++]
#include <iostream>
#include <string>
#include <sstream>
#include <map>
using namespace std;

class Directory {
private: map<string, Directory *> root;
public: Directory() = default;

    Directory *getDir(const string& name) {
        if (root.find(name) != root.end()) return root[name];
        else return createDir(name);
    }
    Directory *createDir(const string& name) {
        root[name] = new Directory();
        return root[name];
    }

    void printTree(const string& separator = "") {
        string tabs = " ";
        tabs += separator;
        map<string, Directory *> contents(root.begin(), root.end());
        for (auto it = contents.begin(); it != contents.end(); it++) {
            cout << separator << it->first << endl;
            it->second->printTree(tabs);
        }
    }
};

int main() {
    int n;
    cin >> n;
    auto *root = new Directory();
    for (int i = 0; i < n; i++) {
        Directory *currentDir = root;
        string fullPath, name;
        cin >> fullPath;
        stringstream ss(fullPath);
        while (getline(ss, name, '\\')) { currentDir = currentDir->getDir(name); }
    }
    root->printTree();
    return 0;
}
\end{lstlisting}
\addsubsection{Объяснение алгоритма}
Алгоритм использует структуру данных hash-map, которая хранит пары ключ-значение, для представления связи вложенности между директориями и воссоздания иерархии. Это позволяет создавать вложенные структуры, имитирующие файловую систему.
Алгоритм работает следующим образом:
\begin{enumerate}
    \item Считывает полный путь к директории.
    \item Разделяет путь на отдельные директории по обратному слешу как по разделителю.
    \item Для каждого имени директории создаёт связь в словаре (метод \verb|getDir|), а если папка не существует, то создает новую (метод \verb|createDir|).
    \item В конце вызывает метод \verb|printTree|, который рекурсивно выводит всю иерархию директорий с отступами вложенности.
\end{enumerate}

\newpage
\addsection{\href{https://acm.timus.ru/problem.aspx?space=1&num=1322}{Задача №1322. Шпион}}
Спецслужбы обнаружили действующего иностранного агента. Шпиона то есть. Установили наблюдение и выяснили, что каждую неделю он через Интернет посылает кому-то странные нечитаемые тексты. Чтобы выяснить, к какой информации получил доступ шпион, требуется расшифровать информацию. Сотрудники спецслужб проникли в квартиру разведчика, изучили шифрующее устройство и выяснили принцип его работы.\\[0.5em]
На вход устройства подается строка текста $S_1 = s_1s_2\dots s_N$. Получив её, устройство строит все циклические перестановки этой строки, то есть $S_2 = s_2s_3\dots s_Ns_1, \dots$, $S_N = s_Ns_1s_2\dots s_{N-1}$. Затем множество строк $S_1, S_2, \dots, S_N$ сортируется лексикографически по возрастанию. И в этом порядке строчки выписываются в столбец, одна под другой. Получается таблица размером $N \times N$. В какой-то строке $K$ этой таблицы находится исходное слово. Номер этой строки вместе с последним столбцом устройство и выдает на выход.\\[0.5em]
Например, если исходное слово $S_1$ = abracadabra, то таблица имеет такой вид:
\begin{multicols}{3}
    \begin{enumerate}
        \item aabracadabr = $S_{11}$
        \item abraabracad = $S_{8}$
        \item abracadabra = $S_{1}$
        \item acadabraabr = $S_{4}$
        \item adabraabrac = $S_{6}$
        \item braabracada = $S_{9}$
        \item bracadabraa = $S_{2}$
        \item cadabraabra = $S_{5}$
        \item dabraabraca = $S_{7}$
        \item raabracadab = $S_{10}$
        \item racadabraab = $S_{3}$
    \end{enumerate}
\end{multicols}
И результатом работы устройства является число $3$ и строка rdarcaaaabb.\\[0.5em]
Это все, что известно про шифрующее устройство. А вот дешифрующего устройства не нашли. Но поскольку заведомо известно, что декодировать информацию можно (а иначе зачем же ее передавать?), Вам предложили помочь в борьбе с хищениями секретов и придумать алгоритм для дешифровки сообщений. А заодно и реализовать дешифратор.\\[1em]
\textbf{Исходные данные:}
\begin{quotebox}
    В первой и второй строках находятся соответственно целое число и строка, возвращаемые шифратором. Длина строки и число не превосходят $10^5$. Строка содержит лишь следующие символы: a-z, A-Z, символ подчеркивания. Других символов в строке нет. Лексикографический порядок на множестве слов задается таким порядком символов:
    ABCDEFGHIJKLMNOPQRSTUVWXYZ\_abcdefghijklmnopqrstuvwxyz\\[0.5em]
    Символы здесь выписаны в порядке возрастания.
\end{quotebox}
\textbf{Результат:}
\begin{quotebox}
    Выведите декодированное сообщение в единственной строке.
\end{quotebox}
\addsubsection{Рабочий код}
\begin{lstlisting}[language=c++]
#include <iostream>
#include <vector>
#include <algorithm>
using namespace std;
int main() {
    int k;
    cin >> k;
    string s;
    cin >> s;
    vector<pair<char, int>> temp(s.size());
    for (int i = 0; i < s.size(); i++) {temp[i] = {s[i], i};}
    std::sort(temp.begin(), temp.end());
    k = k - 1;
    for (int i = 0; i < s.size(); i++) {
        cout << temp[k].first;
        k = temp[k].second;
    }
    return 0;
}
\end{lstlisting}
\addsubsection{Объяснение алгоритма}
Этот код выполняет сортировку символов в строке и затем переставляет их в новом порядке декодированном порядке. Используется структура данных vector для хранения пар символов и их индексов. Алгоритм сортирует эти пары по символам, затем использует второй элемент пары (индекс) для перехода к следующему символу в отсортированной последовательности, начиная с позиции $k-1$. В результате на выходе формируется новая последовательность символов, основанная на исходных позициях символов после сортировки, которая представляет собой декодированное сообщение.
\end{document}