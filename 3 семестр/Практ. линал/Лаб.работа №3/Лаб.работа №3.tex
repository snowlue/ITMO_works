\documentclass[a3paper,14pt]{extarticle}
\usepackage[utf8]{inputenc}
\usepackage{extsizes}
\usepackage[T2A]{fontenc}
\usepackage[english,russian]{babel} 
\usepackage[left=15mm, top=25mm, right=15mm, bottom=30mm, nohead, nofoot]{geometry}
\usepackage{graphicx}  % изобржаения
\usepackage{listings} % листинги (блоки кодов)
\usepackage{xcolor} % определение цветов
\usepackage{nicefrac} % красивые дроби
\usepackage{amsmath,amsfonts,amssymb} % математический пакет
\usepackage{fancybox,fancyhdr} % хедер и футер
\pagestyle{fancy}
\fancyhf{}
\fancyhead[L]{Практическая линейная алгебра}
\fancyhead[R]{Овчинников Павел}
\fancyfoot[C]{\thepage}
\setcounter{page}{1}
\headsep=10mm
\footskip=15mm
\usepackage{hyperref}  % гиперссылки

\definecolor{strings}{rgb}{0,0.6,0}
\definecolor{comments}{rgb}{0,0.3,0}
\definecolor{numbers}{rgb}{0.5,0.5,0.5}
\definecolor{keywords}{rgb}{0.09,0.61,0.95}
\definecolor{background}{rgb}{0.97,0.97,0.97}
\definecolor{urlcolor}{HTML}{3454D1}

\hypersetup{pdfstartview=FitH, linkcolor=linkcolor,urlcolor=urlcolor, colorlinks=true}

\lstdefinestyle{codestyle}{
    backgroundcolor=\color{background},
    commentstyle=\color{comments},
    keywordstyle=\color{keywords},
    stringstyle=\color{strings},
    numberstyle=\tiny\color{numbers},
    basicstyle=\ttfamily\footnotesize,
    breakatwhitespace=false,
    breaklines=true,
    captionpos=b,
    keepspaces=true,
    numbers=left,
    numbersep=5pt,
    showspaces=false,
    showstringspaces=false,
    showtabs=false,
    tabsize=2
}

\lstset{style=codestyle}

\newcommand\NB{\textbf{N\kern-0.32em\textcolor{red}{B}}}

\begin{document}
\section*{\centering Лабораторная работа №3}
\subsection*{\centering Задание №1. Инспектор Гаджет исследует код}
Давайте посмотрим на код и разберём подробно, как он работает:
\begin{lstlisting}[language=Matlab]
verticesCube = [
    -1, 1, 1,-1,-1, 1, 1,-1;
    -1,-1, 1, 1,-1,-1, 1, 1;
    -1,-1,-1,-1, 1, 1, 1, 1;
     1, 1, 1, 1, 1, 1, 1, 1
];
\end{lstlisting}
В массиве перед нами восемь столбцов и четыре строки. Это означает, что вершины куба располагаются в массиве как вертикальные векторы, т.е. сначала в строку пишутся $x$-координаты, затем --- $y$, потом --- $z$ и, наконец, в последней строке пишется для каждого вектора некий параметр $w$. В рамках курса Learn OpenGL, ссылка на который представлена в методическом пособии к лабораторной работе, рассказывается, что компонента $w$ --- параметр однородных (гомогенных) координат. Благодаря нему разница между точками и векторами практически стирается, и мы можем прибавлять к вектору константы при помощи матрицы преобразования, чего не получилось бы сделать без $w$-компоненты.\\
Заметим, что $w$ у всех векторов равна $1$ --- именно поэтому мы часто опускаем эту компоненту, т.к. преобразование четырёхкомпонентного вектора $[x, y, z, w]$ в трёхкомпонентный выглядит как $[\nicefrac{x}{w}, \nicefrac{y}{w}, \nicefrac{z}{w}]$.\\[1em]
Рассмотрим массив граней:
\begin{lstlisting}[language=Matlab, firstnumber=8]
facesCube = [
    1, 2, 6, 5;
    2, 3, 7, 6;
    3, 4, 8, 7;
    4, 1, 5, 8;
    1, 2, 3, 4;
    5, 6, 7, 8
];
\end{lstlisting}
Здесь грани отображены как горизонтальный набор точек, пронумерованных от 1 до 8 по количеству точек у куба. Всего граней у куба шесть, поэтому и строк в массиве ровно столько же.\\[1em] 
Теперь глянем на магию, которая отрисовывает кубик с осями и делает всё красиво:
\begin{lstlisting}[language=Matlab, firstnumber=17]
DrawShape(varticesCube, facesCube, 'blue')
axis equal;
view(3);

function DrawShape(vertices, faces, color)
    patch('Vertices', (vertices(1:3,:)./vertices(4,:))', 'Faces', faces, 'FaceColor', color);
end
\end{lstlisting}
Итак, здесь вызывается функция \verb|DrawShape|, которая отрисовывается все грани куба синим цветом в координатной плоскости. Далее пришлось поэкспериментировать, убрав команды \verb|axis equal| и \verb|view(3)|, чтобы выяснить, за что они отвечают: \verb|axis equal| масштабирует оси соразмерно друг другу, чтобы одна ось не была растянута сильнее, чем друга, а \verb|view(3)| подсказывает Matlab, что мы работаем в 3D, а не 2D (который установлен в программе по умолчанию).\\[1em]
\textbf{\textit{Рубрика «наблюдения»}:} посмотрите-ка, здесь мы в \verb|patch| передаём вершины, грани и цвет граней --- передаются только первые три координаты каждой из вершин, разделённые на четвёртую. А ведь это то, что мы обсуждали выше о том, как четырёхкомпонентный вектор преобразовывывается в трёхкомпонентный.\\[1em]
Мы можем построить любую другую фигуру, изменив как координаты точек, на которых будет строится фигура, так и порядок и количество точек в гранях, по котором сами грани будут строиться.

\subsection*{\centering Задание №2. Wide Putin Walking}
\NB: все дальнейшие преобразования, необходимые для выполнения этого и других заданий, будут сделаны с использованием библиотеки \verb|manim| в \textbf{Python} --- хоть и матрицы преобразования будут представлены как четырёхкомпонентные, на самом же деле в коде будут использоваться трёхкомпонентные с учётом $w$.\\[1.5em]
Первое преобразование, которое нам предстоит сделать в пространстве --- масштабирование или, по-другому говоря, умножение каждой из координат вектора на константы. К примеру, увеличим $x$- и $y$-координаты в 4 раза, а $z$-координату уменьшим в 2 раза. Для этого нам подойдёт следующая матрица $A$, которую мы сразу же и проверим, применив её к вектору $[x, y, z, w]$:
$$Av = \begin{bmatrix}
    4 & 0 & 0 & 0 \\ 0 & 4 & 0 & 0 \\ 0 & 0 & 0.5 & 0 \\ 0 & 0 & 0 & 1
\end{bmatrix}\begin{bmatrix}
    x \\ y \\ z \\ w
\end{bmatrix} = \begin{bmatrix}
    4x + 0y + 0z + 0w \\ 0x + 4y + 0z + 0w \\ 0x + 0y + 0.5z + 0w \\ 0x + 0y + 0z + 1w
\end{bmatrix} = \begin{bmatrix}
    4x \\ 4y \\ 0.5z \\ w
\end{bmatrix}$$
Таким нетрудным телодвижением мы проверили, что координаты действительно ведут себя именно так, как мы хотим, а $w$-компонента остаётся нетронутой и в принципе всегда фиксирована в ходе любых трансформаций вектора. Посмотреть анимашку для такого преобразования можно \href{https://disk.yandex.ru/i/VmQDa1OhRCMm7w}{здесь} $\leftarrow$

\subsection*{\centering Задание №3. Outstanding move}
Теперь пришла очередь уникального в своём роде преобразования пространства --- речь о перемещении. Элементы матрицы умножаясь и складываясь поэлементно, образуют сумму $\alpha x + \beta y + \gamma z$ с некоторыми коэффициентами перед базисными векторами. Эта сумма, как вы видите, складывается из координат исходного вектора с некоторыми коэффициентом --- у нас нет независимого элемента, который мог бы прибавиться ко всем координатам, не домножив их на коэффициенты. Учитывая, что получить в произведении сумму можно только поэлементным сложением, то нам не остаётся ничего, кроме как добавить $w$-компоненту в векторы --- вот откуда она появилась и вот зачем она нужна.\\[0.5em]
Гораздо интереснее вопрос: как в таком случае поведут себя базисные вектора $\vec{i}$, $\vec{j}$, $\vec{k}$, которые расположены на осях координат, и сдвинутся ли они вместе с кубиком? Но перед ответом на этот вопрос, мы посмотрим, как выглядит матрица $B$ такого преобразования и узнаем при помощи умножения, почему она выглядит именно так:
$$Bv = \begin{bmatrix}
    1 & 0 & 0 & 3 \\ 0 & 1 & 0 & -2 \\ 0 & 0 & 1 & -1 \\ 0 & 0 & 0 & 1
\end{bmatrix}\begin{bmatrix}
    x \\ y \\ z \\ w
\end{bmatrix} = \begin{bmatrix}
    1x + 0y + 0z + 3w \\ 0x + 1y + 0z -2w \\ 0x + 0y + 1z -1w \\ 0x+0y+0z+1w
\end{bmatrix} = \begin{bmatrix}
    x + 3w \\ y - 2w \\ z - w \\ w
\end{bmatrix}$$
И вновь компонента $w$ осталась неизменная в преобразованном векторе, а вот к координатам эта компонента прибавилась с неким коэффициентом --- это объясняет, почему матрица выглядит так. Диагональ сохраняет элементы вектора неизменными, а последний столбец добавляет к каждой из координат известную компоненту.\\[0.5em]
А что происходит с базисными векторами? Давайте рассмотрим каждый из них, просто заменив соответствующие неизвестные в преобразованном векторе:
$$\vec{i}^* = \begin{bmatrix}
    4 \\ 0 \\ 0 \\ 1
\end{bmatrix}\qquad\vec{j}^* = \begin{bmatrix}
    0 \\ -1 \\ 0 \\ 1
\end{bmatrix}\qquad\vec{k}^* = \begin{bmatrix}
    0 \\ 0 \\ 0 \\ 1
\end{bmatrix}$$
С одной стороны может показаться, что вектор $\vec{k}$ после преобразования матрицей схлопнулся в точку, но это не так. Убедиться в этом можно, посмотрев мою \href{https://disk.yandex.ru/i/xr_fvZ_HHKuUVA}{анимашку} $\leftarrow$\\[0.5em]
Мы наблюдаем, как вектор $\vec{k}$ оказывается в плоскости $XOY$. Появилось ли у вас в таком случае понимание, почему значение вектора после преобразования оказалось нулевое? Я сделал важное замечание в начале своего отчёта --- при добавлении $w$-компоненты разница между векторами и точками практически стирается. Т.е. фактически векторы перенесли свои концы (точки, прочно связанные с кубом) сообразно с движением куба.\\[0.5em]
Но точки не могут быть базисами, поэтому, чтобы они ими стали, векторы забывают о том, что они в трёхмерном пространстве и действуют так, будто перед ними только их ось. Если взглянуть внимательно на концы векторов после преобразование, то у вектора $\vec{i}$ по оси $x$ значение $4$, у вектора $\vec{j}$ по оси $y$ значение $-1$, а у вектора $\vec{k}$ по оси $z$ как раз значение $0$. Надеюсь, теперь картина прояснилась ;)
\begin{center}
    ***
\end{center}
Теперь попробуем скомбинировать преобразования. Сначала посмотрим \href{https://disk.yandex.ru/i/LH7VCjDJ-MA91Q}{здесь}, как действуют последовательно перемещение и масштабирование, а затем поменяем преобразования местами, выполнив сначала масштабирование, а затем перемещение, и посмотрим на эту красоту \href{https://disk.yandex.ru/i/lP76mzPYP1p4Iw}{здесь}.\\[0.5em]
Причина, по которой результаты различных комбинаций преобразований не сходятся и в первом случае фигура улетает далеко от координатной сетки, а во втором случае практически остаётся на месте, связана с порядком операций. Рассмотрим масштабирование как умножение координаты на $\alpha$, а перемещение, как сумму координаты и $\beta$. И вот как влияет порядок операций:
\begin{center}
    $x \ \rightarrow\ \alpha x \ \rightarrow\ \alpha x + \beta$ \\
    $x \ \rightarrow\ x + \beta \ \rightarrow\ \alpha (x + \beta) = \alpha x + \alpha\beta$
\end{center}
Во втором случае перемещение также расширилось под действием масштабирования. Поэтому порядок применения матриц преобразования важен. \pagebreak

\subsection*{\centering Задание №4. Вот это поворот!}
Матрицы поворота в 3D неким образом схожи с матрицами поворота в 2D --- здесь тоже фигурируют косинусы и синусы. Опять придётся запоминать, где что ставится, да?.. Если в плоскости возможно вращение вокруг точки, то индуктивно из этого следует, что в пространстве вращение происходит вокруг прямых (или осей в нашем случае). Итак, независимых возможных вращений в 3D пространстве ровно три и им соответствуют некоторые матрицы, в которых угол $\theta$ --- угол поворота.\\[0.5em]
Повороту вокруг оси $OX$ соответствует матрица:
$$\begin{bmatrix}
    1 & 0 & 0 & 0 \\ 0 & \cos\theta & -\sin\theta & 0 \\ 0 & \sin\theta & \cos\theta & 0 \\ 0 & 0 & 0 & 1
\end{bmatrix}\begin{bmatrix}
    x \\ y \\ z \\ w
\end{bmatrix} = \begin{bmatrix}
    x \\ y\cos\theta - z\sin\theta \\ y\sin\theta + z\cos\theta \\ w
\end{bmatrix}$$
Повороту вокруг оси $OZ$ соответствует матрица:
$$\begin{bmatrix}
    \cos\theta & -\sin\theta & 0 & 0 \\ \sin\theta & \cos\theta & 0 & 0 \\ 0 & 0 & 1 & 0 \\ 0 & 0 & 0 & 1
\end{bmatrix}\begin{bmatrix}
    x \\ y \\ z \\ w
\end{bmatrix} = \begin{bmatrix}
    x\cos\theta - y\sin\theta\\ x\sin\theta + y\cos\theta \\ z \\ w
\end{bmatrix}$$
Иной студент впал бы в ступор при поиске матрицы для поворота вокруг оси $OY$, но здесь тоже ничего сложного:
$$\begin{bmatrix}
    \cos\theta & 0 & \sin\theta & 0 \\ 0 & 1 & 0 & 0 \\ -\sin\theta & 0 & \cos\theta & 0 \\ 0 & 0 & 0 & 1
\end{bmatrix}\begin{bmatrix}
    x \\ y \\ z \\ w
\end{bmatrix} = \begin{bmatrix}
    x\cos\theta + z\sin\theta\\ y \\ z\cos\theta - x\sin\theta \\ w
\end{bmatrix}$$
Проверим, обладают ли повороты в 3D свойством коммутативности, и заодно посмотрим, как меняется кубик в зависимости от выбранной комбинации матриц --- анимашку можно глянуть \href{https://disk.yandex.ru/i/BNLJQFTG22JTfQ}{здесь} и \href{https://disk.yandex.ru/i/YEaJCiizGnh7AA}{здесь}. Приходим к выводу, что матрицы поворотов не коммутативны, в отличие от матриц поворота в 2D. Это связано с тем, что независимых поворотов в 3D три, а в 2D всего лишь один.

\subsection*{\centering Задание №5. You spin my head round...}
Чтобы вращение кубика происхоило около одной вершины, необходимо её как-то зафиксировать. Для этого спросим себя: какая из точек при повороте пространства остаётся фиксированной? Конечно же, нулевая ;)\\[0.5em]
Остаётся просто закрепить один из углов куба преобразованием перемещения в точке $(0, 0, 0)$ и выполнить поворот. Итак, вот какие матрицы мы будем применять:
$$C = \begin{bmatrix}
    1 & 0 & 0 & 1 \\ 0 & 1 & 0 & 1 \\ 0 & 0 & 1 & 1 \\ 0 & 0 & 0 & 1
\end{bmatrix}$$
Эта матрица сдвинет нижний угол куба со стороной равной 2 в точку начала координат.
$$D = \begin{bmatrix}
    \nicefrac{1}{2} & \nicefrac{-\sqrt{3}}{4} & \nicefrac{-3}{4} & 0 \\
    \nicefrac{\sqrt{3}}{2} & \nicefrac{1}{4} & \nicefrac{\sqrt{3}}{4} & 0 \\
    0 & \nicefrac{-\sqrt{3}}{2} & \nicefrac{1}{2} & 0 \\ 0 & 0 & 0 & 1
\end{bmatrix}$$
А эта матрица совершит поворот на $60^\circ$ вокруг осей $OX$ и $OY$ одновременно. Давайте посмотрим, что получилось --- \href{https://disk.yandex.ru/i/i6hjPyfOMITk6A}{анимашка уже готова} :)

\subsection*{\centering Задание №6. Свет, камера, мотор}
\textit{Студент увидел слово MATLAB и упал в обморок... Мы его долго откачивали, но он сказал, что у него он не установлен, а там 20 Гб --- они же будут целую вечность качаться :/$\qquad\qquad\qquad\quad$А камеры в }\verb|manim|\textit{ устроены непросто...}
\end{document}