\documentclass[a3paper,14pt]{extarticle}
\usepackage[utf8]{inputenc}
\usepackage{extsizes}
\usepackage[T2A]{fontenc}
\usepackage[english,russian]{babel} 
\usepackage[left=15mm, top=25mm, right=15mm, bottom=25mm, nohead, nofoot]{geometry}
\usepackage{amsmath,amsfonts,amssymb} % математический пакет
\usepackage{fancybox,fancyhdr} % хедер и футер
\pagestyle{fancy}
\fancyhf{}
\fancyhead[L]{Практическая линейная алгебра}
\fancyhead[R]{Овчинников Павел}
\fancyfoot[C]{\thepage}
\setcounter{page}{1}
\headsep=10mm 
\usepackage{hyperref}
% \usepackage{xcolor}
% \hypersetup{colorlinks=true, allcolors=[RGB]{010 090 200}} % цвет ссылок 
% \newcommand{\lr}[1]{\left({#1}\right)} % команда для скобок

\begin{document}
\subsection*{\centering Задание №1}
Найдём собственные числа матрицы:
$$\det(A-\lambda E) = \begin{vmatrix}
    5-\lambda & -7 & -4 \\ 0 & -2-\lambda & 0 \\ 0 & 0 & 1-\lambda
\end{vmatrix} = (5-\lambda)(-2-\lambda)(1-\lambda)=0$$
$$\lambda_1 = 5 \qquad \lambda_2 = -2 \qquad \lambda_3 = 1$$
Теперь найдём собственные вектора каждого из собственный чисел:
$$\lambda_1 = 5$$
$$\begin{bmatrix}
    5 & -7 & -4 \\ 0 & -2 & 0 \\ 0 & 0 & 1
\end{bmatrix}\begin{bmatrix}
    x \\ y \\ z
\end{bmatrix} = \begin{bmatrix}
    5x \\ 5y \\ 5z
\end{bmatrix}$$
$$\begin{cases}
    5x-7y-4z=5x \\
    -2y = 5y \\
    z = 5z
\end{cases} \Leftrightarrow \begin{cases}
    5x-7y-4z=5x \\
    y = 0 \\
    z = 0
\end{cases} \Leftrightarrow \begin{cases}
    x = x \\
    y = 0 \\
    z = 0
\end{cases} \Rightarrow v_1 = \begin{pmatrix}
    1 \\ 0 \\ 0
\end{pmatrix}$$ \,\\
$$\lambda_2 = -2$$
$$\begin{bmatrix}
    5 & -7 & -4 \\ 0 & -2 & 0 \\ 0 & 0 & 1
\end{bmatrix}\begin{bmatrix}
    x \\ y \\ z
\end{bmatrix} = \begin{bmatrix}
    -2x \\ -2y \\ -2z
\end{bmatrix}$$
$$\begin{cases}
    5x-7y-4z=-2x \\
    -2y = -2y \\
    z = -2z
\end{cases} \Leftrightarrow \begin{cases}
    5x-7y-0=-2x \\
    y = y \\
    z = 0
\end{cases} \Leftrightarrow \begin{cases}
    x = y \\
    y = y \\
    z = 0
\end{cases} \Rightarrow v_2 = \begin{pmatrix}
    1 \\ 1 \\ 0
\end{pmatrix}$$\,\\
$$\lambda_3 = 1$$
$$\begin{bmatrix}
    5 & -7 & -4 \\ 0 & -2 & 0 \\ 0 & 0 & 1
\end{bmatrix}\begin{bmatrix}
    x \\ y \\ z
\end{bmatrix} = \begin{bmatrix}
    x \\ y \\ z
\end{bmatrix}$$
$$\begin{cases}
    5x-7y-4z=x \\
    -2y = y \\
    z = z
\end{cases} \Leftrightarrow \begin{cases}
    5x-0-4z=x \\
    y = 0 \\
    z = z
\end{cases} \Leftrightarrow \begin{cases}
    x = z \\
    y = 0 \\
    z = z
\end{cases} \Rightarrow v_3 = \begin{pmatrix}
    1 \\ 0 \\ 1
\end{pmatrix}$$ \, \\[10mm]
Спектральное разложение --- разожение, которое представляет матрицу в базисе собственных векторов, где она представлена диагональной матрицей с собственными числами на диагонали. И вот как оно выглядит:
$$A = PA^*P^{-1}=\begin{bmatrix}
    1 & 1 & 1 \\ 0 & 1 & 0 \\ 0 & 0 & 1
\end{bmatrix}\begin{bmatrix}
    5 & 0 & 0 \\ 0 & -2 & 0 \\ 0 & 0 & 1
\end{bmatrix}\begin{bmatrix}
    1 & 1 & 1 \\ 0 & 1 & 0 \\ 0 & 0 & 1
\end{bmatrix}^{-1}=\begin{bmatrix}
    1 & 1 & 1 \\ 0 & 1 & 0 \\ 0 & 0 & 1
\end{bmatrix}\begin{bmatrix}
    5 & 0 & 0 \\ 0 & -2 & 0 \\ 0 & 0 & 1
\end{bmatrix}\begin{bmatrix}
    0 & 0 & 1 \\ 0 & 1 & 0 \\ 1 & -1 & -1
\end{bmatrix}$$\pagebreak
\subsection*{\centering Задание №2}
Найдём собственные числа матрицы:
$$\det(A-\lambda E) = \begin{vmatrix}
    2-\lambda & 0 & -1 \\ 1 & 2-\lambda & -1 \\ 0 & 0 & 1-\lambda
\end{vmatrix} = (2-\lambda)(2-\lambda)(1-\lambda)=0$$
$$\lambda_{1,2} = 2 \qquad \lambda_3 = 1$$
Теперь найдём собственные вектора каждого из собственный чисел:
$$\lambda_{1,2} = 2$$
$$\begin{bmatrix}
    2 & 0 & -1 \\ 1 & 2 & -1 \\ 0 & 0 & 1
\end{bmatrix}\begin{bmatrix}
    x \\ y \\ z
\end{bmatrix} = \begin{bmatrix}
    2x \\ 2y \\ 2z
\end{bmatrix}$$
$$\begin{cases}
    2x-z=2x \\
    x+2y-z=2y \\
    z = 2z
\end{cases} \Leftrightarrow \begin{cases}
    2x-z=2x \\
    x+2y-z=2y \\
    z = 0
\end{cases} \Leftrightarrow \begin{cases}
    2x=2x \\
    x = 0 \\
    z = 0
\end{cases} \Rightarrow v_1 = \begin{pmatrix}
    0 \\ 1 \\ 0
\end{pmatrix}$$
Найдём присоединённый вектор к $v_1$:
$$\begin{bmatrix}
    0 & 0 & -1 \\ 1 & 0 & -1 \\ 0 & 0 & -1
\end{bmatrix}\begin{bmatrix}
    x \\ y \\ z
\end{bmatrix} = \begin{bmatrix}
    0 \\ 1 \\ 0
\end{bmatrix}$$
$$\begin{cases}
    z=0 \\
    x-z=1 \\
    z = 0
\end{cases} \Leftrightarrow \begin{cases}
    z=0 \\
    x=1 \\
    z=0
\end{cases} \Rightarrow w_1 = \begin{pmatrix}
    1 \\ 1 \\ 0
\end{pmatrix}$$\,\\
$$\lambda_3 = 1$$
$$\begin{bmatrix}
    2 & 0 & -1 \\ 1 & 2 & -1 \\ 0 & 0 & 1
\end{bmatrix}\begin{bmatrix}
    x \\ y \\ z
\end{bmatrix} = \begin{bmatrix}
    x \\ y \\ z
\end{bmatrix}$$
$$\begin{cases}
    2x-z=x \\
    x+2y-z=y \\
    z = z
\end{cases} \Leftrightarrow \begin{cases}
    x=z \\
    2y = y \\
    z = z
\end{cases} \Rightarrow v_2 = \begin{pmatrix}
    1 \\ 0 \\ 1
\end{pmatrix}$$\,\\[10mm]
Жорданово разложение --- разложение матрицы, образованное Жордановой формой матрицы в базисе её собственных и присоединённых векторов. Она выглядит так:
$$A=P\mathcal{J}P^{-1}=\begin{bmatrix}
    0 & 1 & 1 \\ 1 & 1 & 0 \\ 0 & 0 & 1
\end{bmatrix}\begin{bmatrix}
    2 & 1 & 0 \\ 0 & 2 & 0 \\ 0 & 0 & 1
\end{bmatrix}\begin{bmatrix}
    0 & 1 & 1 \\ 1 & 1 & 0 \\ 0 & 0 & 1
\end{bmatrix}^{-1} = \begin{bmatrix}
    0 & 1 & 1 \\ 1 & 1 & 0 \\ 0 & 0 & 1
\end{bmatrix}\begin{bmatrix}
    2 & 1 & 0 \\ 0 & 2 & 0 \\ 0 & 0 & 1
\end{bmatrix}\begin{bmatrix}
    -1 & 1 & 1 \\ 1 & 0 & -1 \\ 0 & 0 & 1
\end{bmatrix}$$
\end{document}