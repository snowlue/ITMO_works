\documentclass[a3paper,14pt]{article}
\usepackage[utf8]{inputenc}
\usepackage{extsizes}
\usepackage[T2A]{fontenc}
\usepackage[english,russian]{babel} 
\usepackage[left=15mm, top=25mm, right=15mm, bottom=25mm, nohead, nofoot]{geometry}
\usepackage{graphicx}
\usepackage{nicefrac}
\usepackage{tikz}
\usepackage{amsmath,amsfonts,amssymb} % математический пакет
\usepackage{fancybox,fancyhdr} % хедер и футер
\pagestyle{fancy}
\fancyhf{}
\fancyhead[L]{Практика №5}
\fancyhead[R]{Дифференциальные уравнения}
\fancyfoot[C]{\thepage}
\setcounter{page}{1}
\headsep=10mm 
\usepackage{hyperref}
\usepackage{cancel}

\newlength{\tempheight}
\newcommand{\Let}[0]{
\mathbin{\text{\settoheight{\tempheight}{\mathstrut}\raisebox{0.4\pgflinewidth}{
\tikz[baseline=0.5ex,line cap=round,line join=round] \draw (0,0) --++ (0.3em,0) --++ (0,2.3ex) --++ (-0.3em,0);
}}}}
\newcommand*\squared[1]{\tikz[baseline=(char.base)]{
            \node[shape=rectangle,draw,inner sep=4pt] (char) {$#1$};}}
\newcommand{\at}{\biggr\rvert}

\begin{document}
\subsection*{\centering Линейные однородные уравнения высшего порядка}
\noindent1. $y'' - 5y -6y = 0$\\
Заменяем все производные на соответствующие $r$ --- таким образом составляем характеристическое уравнение:
$$r^2-5r-6=0$$
$$\left[\begin{array}{l}
    r_1 = 6 \\ r_2 = -1
\end{array}\right.\quad r_1,r_2 \in \mathbb{R},\quad k = 1$$
$$y = C_1e^{-x}+C+2e^{6x}$$\,\\[0.5em]
2. $y'''-6y''+13y'=0$
$$r^3-6r^2+13r=0$$
$$r(r^2-6r+13)=0$$
$$\left[\begin{array}{l}
    r_1 = 0 \in \mathbb{R},\quad k=1 \\ r_{2,3} = 3 \pm 2i \in \mathbb{C},\quad k=1
\end{array}\right.$$
$$y=C_1e^{0\cdot x}+e^{3x}(C_2\cos2x+C_3\sin2x) = C_1 + e^3x(C_2\cos2x+C_3\sin2x)$$\,\\[0.5em]
3. $y''+4y'+4y=0$
$$r^2+4r+4 = 0$$
$$(r+2)^2 = 0$$
$$r = -2 \in \mathbb{R},\quad k=2$$
$$y = e^{-2x}(C_1x+C_2)$$\,\\[0.5em]
4. $y^{(7)}+2y^{(5)}+y^{(3)}$
$$r^7+2r^5+r^3=0$$
$$r^3(r^4+2r^2+1)=0$$
$$r^3(r^2 + 1)^2=0$$
$$\left[\begin{array}{l}
    r_1 = 0 \in \mathbb{R},\quad k = 3 \\
    r_{2,3}=\pm i \in \mathbb{C},\quad k = 2
\end{array}\right.$$
$$y=C_1e^{0\cdot x}(C_1x^2 + C_2x + C_3) + e^{0\cdot x}((C_4x + C_5)\cos x + (C_6x+C_7)\sin x) =$$
$$= C_1x^2 + C_2x + C_3 + (C_4x + C_5)\cos x + (C_6x+C_7)\sin x$$\,\\[0.5em]
5. $y''+4y'+5y = 0\qquad y(0) = -3;\ \ y'(0) = 0$
$$r^2+4r+5=0$$
$$r_{1, 2} = -2 \pm i,\quad k=2 $$
$$y = e^{-2x}(C_1\cos x+C_2\sin x)$$
$$y' = -2e^{-2x}(C_1\cos x+C_2\sin x)+e^{-2x}(-C_1\sin_x+C_2\cos x)=e^{-2x}((-2C_1+C_2)\cos x-(C_1+2C_2)\sin x)$$
$$y(0) =-3 \ \Rightarrow\  C_1 = -3$$
$$y'(0)=0 \ \Rightarrow\  -2C_1+C_2 = 0 \ \Rightarrow\  C_2 = -6$$
$$\text{Получаем с учётом констант: } y = -e^{-2x}(3\cos x + 6\sin x)$$


\subsection*{\centering Линейные неоднородные уравнения высшего порядка}
В общем виде:
\begin{itemize}
    \item $n = 2$ (второй порядок), $r_1, r_2$ --- корни ХУ (характеристического уравнения): \\ $y_1 = e^{r_1x}\int e^{(r_2-r_1)x}(\int qe^{-r_2x}dx)dx$
    \item $n = 3$, $r_1, r_2, r_3$ --- корни ХУ (характеристического уравнения): \\ $y_1 = e^{r_1x}\int e^{(r_2-r_1)x}(\int e^{(r_3-r_2)x}(\int qe^{-r_3x}dx)dx)dx$
\end{itemize}
1. $y''+6y'+5y=25x^2-2$\\
1) Решаем ЛОУ:
$$r^2+6r+5=0$$
$$\left[\begin{array}{l}
    r_1 = -1 \\ r_2 = -5
\end{array}\right.\quad r_{1,2}\in\mathbb{R},\quad k = 1$$
$$y = C_1e^{-5x}+C_2e^{-x}$$
2) Решение для остаточного члена: $q = 25x^2-2=e^{0\cdot x}(25x^2-2)\quad m=0$ не корень ХУ
$$y_1 \sim q(x)\quad y_1=Ax^2+Bx+C$$
$$y_1'=2Ax+B\quad y_1''=2A$$
$$2A+6(2Ax+B)+5(Ax^2+Bx+C)=25x^2-2$$
$$5Ax^2+(12A+5B)x+(2A+6B+5C)=25x^2-2$$
$$\begin{cases}
    5A = 25 \\
    12A+5B = 0 \\
    2A+6B+5C=-2
\end{cases} \Leftrightarrow \begin{cases}
    A = 5 \\
    B = -12 \\
    C = 12
\end{cases}\qquad y_1=5x^2-12x+12$$
$$\text{Общее решение: } y = u+y_1 = C_1e^{-5x}+C_2e^{-x}+5x^2-12x+12$$\,\\[0.5em]
2. $y'''+4y'=8e^{2x}+5e^x\sin x$ \\
1) ЛОУ
$$r^3+4r = 0$$
$$r(r^2+4) = 0$$
$$\left[\begin{array}{l}
    r_1=0 \\ r_{2,3}=\pm 2i
\end{array}\right.$$
$$u = C_1+C_2\cos 2x + C_3 \sin 2x$$
2) ЛНУ: $q = 8 e^{2x}+5e^x\sin x\quad m =2$ не корень ХУ
$$y_1=e^{2x}A+e^x(B\cos x+ C\sin x)$$
$$y_1'=2Ae^{2x}+e^x(B\cos x+C\sin x) +e^x(C\cos x -B\sin x)=2Ae^{2x}+e^x((B+C)\cos x +(C-B)\sin x)$$
$$y_1''=4Ae^{2x}+e^x((B+C)\cos x +(C-B)\sin x)+e^x((C-B)\cos x - (B+C)\sin x) = =4Ae^{2x}+e^x(2C\cos x -2B\sin x)$$
$$y_1'''=8Ae^{2x}+e^x(2C\cos x -2B\sin x)+e^x(-2C\sin x - 2B\cos x) = 8Ae^{2x}+2e^x((C-B)\cos x - (B+C)\sin x)$$
$$8Ae^{2x}+2e^x((C-B)\cos x - (B+C)\sin x)+4(=2Ae^{2x}+e^x((B+C)\cos x +(C-B)\sin x))= 8 e^{2x}+5e^x\sin x$$
$$\begin{cases}
    16A = 8 \\
    2B+6C = 0 \\
    2C-6B = 5
\end{cases}\Leftrightarrow \begin{cases}
    A = \nicefrac{1}{2} \\
    B = \nicefrac{-3}{4} \\
    C = \nicefrac{1}{4}
\end{cases}$$
$$y_1 = \frac{1}{2}e^{2x}+e^x\left(-\frac{3}{4}\cos x+\frac{1}{4}\sin x\right)$$
$$y = C_1+C_2\cos 2x+C_3\sin2x + \frac{1}{2}e^{2x}+e^x\left(\frac{1}{4}\sin x-\frac{3}{4}\cos x\right)$$
\end{document}